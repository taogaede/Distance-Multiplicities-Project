\documentclass[12]{article}

\usepackage{geometry}
\usepackage{amsmath, amsthm, amssymb}
\usepackage{graphicx}
\usepackage{tikz}
\usepackage{booktabs} % See the package documentation for guidelines on formal tables: https://ctan.org/pkg/booktabs
\usepackage{verbatim} % Used to typeset, for example, code snippets or pseudo-code for algorithms.
\usepackage{dsfont} % Extra fontset for helpful mathematics symbols, e.g. \mathds{1}
\usepackage{etoolbox} % Used to allow boolean variables for use in the title page
\usepackage{import}
\usepackage{lipsum}
\usepackage{subcaption}
\usepackage{float}
\usepackage{enumitem}
\usepackage{tabularx}
\usepackage{array}
\usepackage{pdfpages}
\usepackage{mathtools}
\usepackage{hyperref}
\newcolumntype{C}[1]{>{\centering\arraybackslash}m{#1}}
\newcommand{\R}{\mathbb{R}}
\newcommand{\Q}{\mathbb{Q}}
\newcommand{\C}{\mathbb{C}}
\newcommand{\N}{\mathbb{N}}
\newcommand{\Z}{\mathbb{Z}}
\newcommand{\T}{\mathbb{T}}
\newcommand{\cA}{\mathcal{A}}
\newcommand{\cB}{\mathcal{B}}
\newcommand{\cD}{\mathcal{D}}
\newcommand{\cP}{\mathcal{P}}
\newcommand{\cM}{\mathcal{M}}
\newcommand{\abs}[1]{\left\lvert #1 \right\rvert}
\newcommand{\norm}[1]{\left\lVert #1 \right\rVert}
\newcommand{\set}[2]{\left\{#1 \ : \ #2\right\}}
\newcommand{\conv}[1]{\underset{#1}\longrightarrow}
\newcommand{\Mod}[1]{\ (\mathrm{mod}\ #1)}
\newcommand{\Supp}[0]{\ \mathrm{Supp}\ }
\DeclarePairedDelimiter\ceil{\lceil}{\rceil}
\DeclarePairedDelimiter\floor{\lfloor}{\rfloor}
\DeclareMathOperator{\lcm}{lcm}
\newcommand{\Cross}{\mathbin{\tikz [x=1.4ex,y=1.4ex,line width=.2ex] \draw (0,0) -- (1,1) (0,1) -- (1,0);}}

\newcommand\restr[2]{{% we make the whole thing an ordinary symbol
		\left.\kern-\nulldelimiterspace % automatically resize the bar with \right
		#1 % the function
		\vphantom{\big|} % pretend it's a little taller at normal size
		\right|_{#2} % this is the delimiter
}}
% Custom math operators (analogous to \lim, \sup, etc).
\DeclareMathOperator{\id}{id}
\DeclareMathOperator{\subspan}{span}
\DeclareMathOperator{\sgn}{sgn}
\DeclareMathOperator{\diam}{Diam}

\newtheorem{thm}{Theorem}[section] % Numbering is impacted by [chapter]; could do [section] or [subsection] also.
\newtheorem{lem}{Lemma} % The [thm] argument says to number Lemma in sequence with Theorem.
\newtheorem{prop}[thm]{Proposition}
\newtheorem{cor}[thm]{Corollary}
\newtheorem{conj}[thm]{Conjecture}
\newtheorem{question}{Question}
% These environments are unnumbered and will not count toward the numbering.
%\newtheorem*{question}{Question}
\newtheorem*{answer}{Answer}
\newtheorem*{conjecture}{Conjecture}
\newtheorem*{claim}{Claim}
% These environments are definitions; they have a different style (bold label, standard font).
\theoremstyle{definition}
\newtheorem{defn}[thm]{Definition} % These definitions are also numbered in sequence with Theorem.
\newtheorem{eg}{Example}
\newtheorem{rem}[thm]{Remark}
\newtheorem{obs}{Observation}

\title{ \vspace{-3cm} Ph.D. Working Notes }
%\author{Tao Gaede}

\begin{document}
	\maketitle
	\tableofcontents
	
	\section{Crescent Labelled Trees}
	
	Let $T$ be a tree of order $n$.  A crescent labelling of $T$ is a map $L: E(T) \mapsto \{1,2, \ldots, t\}$, such that the distance multiset of $L(T)$ is of the form $\{d_1^1,d_2^2, \ldots, d_{n-1}^{n-1}\}$.  The diameter of $T$, denoted $\diam(T)$, is the length of the $(u,v)$-path in $T$.  The max degree of $T$ is denoted $\Delta(T)$.  %We use $\ell$ to denote the number of leaves of $T$.
	

\begin{lem}[Basic Diameter Lower Bound]
	Let $t$ be a positive integer.  If $L(T)$ is a crescent labelling of the tree $T$ with weights $\{1,2,\ldots,t\}$, then $\diam(T) \geq \tfrac{n-1}{t}$.
\end{lem}

\begin{proof}
	Since there are at least $n-1$ distinct distances, there is a distance $d$ with value at least $n-1$.  Let $u,v \in V(T)$ such that $d(u,v) = d$, then since $t$ is the max edge weight, this means that the number of edges on a $(u,v)$-path is at least $\tfrac{d}{t} \geq \tfrac{n-1}{t}$.
\end{proof}
	
	
	For a pair of vertices $u,v \in V(T)$, we denote the $(u,v)$-path in $T$ as $P(u,v)$.  Lemma \ref{Lemma-DegreeClasses} below generalizes the observation underlying the maximum degree upper bound of $\sim$ $\sqrt{2n}$.
	
	\begin{lem}\label{Lemma-DegreeClasses}
		Let $T$ be a tree of order $n$.  For every $i \in [1,n-1]$, $M \in V(T)$, and $j \in \mathcal{N}(M)$, define
		$$D_j := \{u \in V(T) \setminus \{M\}: d(u,M) = d_i, j \in P(u,M)\}.$$
		Then distance $2d_i$ occurs with multiplicity at least $\sum_{j < k}^{\deg(M)} |D_j|\cdot |D_k|$.
	\end{lem}

	\begin{proof}
		Let $M \in V(T)$ and $i \in [1,n-1]$.  Since $T$ is a tree, there is always a unique $(u,v)$-path for all $u,v \in V(T)$.  So, for each $u \in D_j$ and $v \in D_k$, the $(u,v)$-path must go through $M$, which means $d(u,v) = d(u,M) + d(M,v) = 2d_i$.  There are $|D_j| \cdot |D_k|$ such $u$ and $v$ pairs, so indeed $2d_i$ has multiplicity at least $\sum_{j < k}^{\deg(M)} |D_j|\cdot |D_k|$. \qedhere
	\end{proof}
	
	Now we apply the lemma to get a condition on crescent labelled trees.
	
	\begin{prop}[Max Multiplicity Condition]
		Let $L(T)$ be a crescent labelling of a tree $T$.  Then for every $i \in [1,n-1]$, $M \in V(T)$, and $j \in \mathcal{N}(M)$, 
		$$\sum_{j < k}^{\deg(M)} |D_j|\cdot |D_k| < n.$$
	\end{prop}
	
	\begin{proof}
		Since $L(T)$ is a crescent labelling of $T$, no distance can have multiplicity greater than $n-1$and $T$ is a tree.  Since $T$ is a tree, it follows by Lemma \ref{Lemma-DegreeClasses} that for each vertex $M \in V(T)$, $i \in [1,n-1]$, $\sum_{j < k}^{\deg(M)} |D_j|\cdot |D_k| < n$. \qedhere
	\end{proof}
	
	Next is a general lemma on $\{1,2,\ldots,t\}$-words containing subwords with $t-1$ consecutive $1s$.
	
	\begin{lem}[Arithmetic Condition]\label{Lemma-ArithmeticCondition}
		Let $k \geq 2t$.  Let $\mathbf{w}$ be a $\{1,2,\ldots, t\}$-word with length $k$.  If $w_{a-t+2} = w_{a-t+3} = \cdots = w_{a} = 1$ for some $a \in \{1,2, \ldots, k\}$, then each value 
		$$1,2, \ldots, \max\biggr\{\sum_{i = 1}^{a}w_i, \sum_{i=a-t+2}^{k}w_i\biggr\}$$ 
		occurs as a partial sum in $\mathbf{w}$.
	\end{lem}
	
	\begin{proof}
		Suppose without loss of generality that $\sum_{i = 1}^{a}w_i \leq \sum_{i=a-t+2}^{k}w_i$.  Then it is sufficient to show that every value $1,2,\ldots, \sum_{i=a-t+2}^{k}w_i$ occurs as a partial sum in $\mathbf{w}$.  Call $w_{a-t+2},w_{a-t+3}, \ldots, w_a$ the \emph{unit segment} of $\mathbf{w}$ and $w_{a+1},w_{a+2},\ldots,w_k$ the \emph{non-unit segment} of $\mathbf{w}$.  We proceed by induction on the number of terms $r$ in the non-unit segment of $\mathbf{w}$.  When $r = 1$, $w_{a+r} \in \{1,\ldots,t\}$, and since the unit segment has $t-1$ 1s, for each $j \in \{1,2,\ldots,t-1\}$, we have the partial sums $j = \sum_{i=0}^{j-1}w_{a-i}$.  Then the values between $w_{a+r}$ and $\sum_{i=a-t+2}^{a+r}w_i$ are of the form $w_{a+r}+ \sum_{i=0}^{j-1}w_{a-i}$.  For the inductive step, the values $1,2,\ldots,\sum_{i=a-t+2}^{a+r-1}w_i$ occur at least once by inductive hypothesis.  We have that $w_{a+r} \in \{1,2,\ldots,t\}$ and the values between $\sum_{i=a+1}^{a+r-1}w_i$ and $\sum_{i=a+1}^{a+r}w_i$ can be obtained from $\sum_{i=a+1-j}^{a+r-1}w_i$ for each $j \in \{1,2,\ldots,t-1\}$.  Then similarly the values between $\sum_{i=a+1}^{a+r}w_i$ and $\sum_{i=a-t+2}^{a+r}w_i$ are $\sum_{i=a+1-j}^{a+r}w_i$ for $j \in \{1,2,\ldots,t-1\}$. \qedhere
	\end{proof}
	
	We now apply this arithmetic lemma to crescent labelled trees to show that when there are many consecutive 1s on a path, the path cannot be too long with many large weight edges.
	
	\begin{prop}
		Let $L(T)$ be a crescent labelling of a tree $T$ with edge weights in $\{1,2,\ldots,t\}$.  Then for every path $P = (v_1v_2,v_2v_3,\ldots,v_{t-1}v_t)$ in $T$ such that $w(v_iv_{i+1}) = 1$ for $i \in \{1,2,\ldots,t-1\}$, it follows that $\max\{d(v_1,u): u \in V(T)\} < n$ and $\max\{d(v_t,u): u \in V(T)\} < n$.
	\end{prop}
	
	\begin{proof}
		Let $T$ be a tree with a path $P$ specified in the proposition statement and $L(T)$ a crescent labelling.  It is sufficient to show that $\max\{d(v_1,u): u \in V(T)\} < n$ since the case for $v_t$ is similar.  Let $u' \in V(T)$ such that $d(v_1,u') = \max\{d(v_1,u): u \in V(T)\}$.  By Lemma \ref{Lemma-ArithmeticCondition}, every distance $1,2,\ldots, d(v_1,u')$ occurs at least once.  Since $L(T)$ is a crescent labelling, there can be at most $n-1$ distinct distances, so $d(v_1,u') < n$ as desired. \qedhere
	\end{proof}
	The implication for when $t=2$ is quite strong since this imposes a max distance condition on vertices incident to edges with weight $1$.
	\begin{cor}
		Let $L(T)$ be a crescent labelling of a tree $T$.  If $t = 2$, then every vertex incident to an edge with weight $1$ has max distance at most $n-1$.
	\end{cor}
	
	%\begin{prop}
	%	Let $L(T)$ be a crescent labelling of a tree $T$.  For every path $P$ of length $t-1$ whose edges all have weight $1$, each end vertex $v \in P$ satisfies 
	%	$$\max\{d(v,u):u \in V(T)\} \leq n.$$
	%\end{prop}
	
	What follows is a basic lemma about trees that may turn out to be useful in case parameterizing by number of leaves becomes sensible.
	
	\begin{lem}[From Chartrand and Lesniak's text ``Graphs and Digraphs" 4th edition]
		Let $T$ be a tree with $n_i$ vertices with degree $i$, where $i \in \{1,2,\ldots,\Delta(T)\}$.  Then $n_1 = n_3 + 2n_4 + 3n_5 + \cdots + (\Delta(T)-2)n_{\Delta(T)} + 2$.
	\end{lem}
	
	\begin{proof}
		Note that $n = \sum_{i=1}^{\Delta(T)}n_i$.  Since $T$ is a tree, 
		$$\sum_{i=1}^{\Delta(T)} in_i  = \sum_{v \in V(T)} \deg(v) = 2(n-1) = 2\biggr(\sum_{i=1}^{\Delta(T)}n_i \biggr) - 2.$$  
		Rearranging gives $2+ \sum_{i=1}^{\Delta(T)} (i-2)n_i = 0.$
	\end{proof}
	
	\begin{cor}
		If $T$ is a tree, then $\sum_{i=3}^{\Delta(T)}(i-2)n_i < n_1$
	\end{cor}
	
	\section{Polynomial Method}
	
	Let $L(G)$ be a crescent labelling of a graph $G$ with corresponding distance multiset $\{d_1^{1},d_2^{2},\ldots,d_{n-1}^{n-1}\}$. Consider the bipartite multigraph $\mathcal{M} := \mathcal{M}(G)$ where $V(\mathcal{M}) = X \cup Y$, where $X$ consists of the distinct distances $d_1,d_2,\ldots,d_{n-1}$ and $Y$ consists of the vertices of $G$, $v_1,v_2,\ldots,v_n$.  For $d_k\in X$ and $v_i \in Y$, an edge $d_kv_i \in E(\mathcal{M})$ is included for every $j \in [n]$ such that $d(v_i,v_j) = d_k$.  Note that since $L(G)$ is a crescent labelling, for each $k \in [n-1]$, $\deg(d_k) = 2k$.  Observe also that the multiset neighbourhood of $v_i$ is the multiset of the $n-1$ distances between $v_i$ and the other vertices in $G$.
		
	We show a variation of a result from Alon (see proof of Theorem 6.1 in \cite{alon}) about the existence of $p$-regular subgraphs of a multigraph whose average degree is very close to its max degree.  If we relax this strong average degree condition, we can still obtain a rather powerful result whereby a subgraph $U$ of $\mathcal{M}$ has vertex degrees in $\{p,2p,3p,4p,5p,6p,7p\} \cap [2(n-1)]$, where $p \in [\tfrac{n}{4},\tfrac{n}{2}]$, which is significant because $\mathcal{M}$ is bipartite and so the structure of $U$ can tell us some things about how the distances relate to the vertices in $G$.  This subgraph $U$ likely can't be too small, since then there would be a vertex $v \in G$ with too many other vertices at some distance $d$ from $v$.  
	
	\begin{rem}
		Relating the size of this $U$ to the structure of $G$ might be a fruitful way to proceed.  For instance, paths require $|U|$ to be quite large (no vertex is at distance $d$ with more than $2$ other vertices for each $d$).  I think stars might be similar in that they require $|U|$ to be rather large.  Perhaps if $p \sim n/4$, or even asymptotically when $p \sim n/2 - (n/2)^{0.525}$ or so, $U$ being large with min degree $p$ forces convergence of crescent labelled trees to paths and stars.  But I admit, I'm not really sure right now what to do when $|U|$ is big.
	\end{rem}
	
	
	%The key result here is Corollary XX, which says that any crescent labelled graph forces at least one of the vertices to be at distance $d$ from at least $p$ other vertices, where $p \in [\tfrac{n}{4}, \tfrac{n}{2}]$.  
	
	The proof applies Alon's combinatorial nullstellensatz \cite{alon}.  The corollary of the nullstellensatz that we use is as follows:
	
	\begin{lem}[Combinatorial Nullstellensatz]\label{Lemma-CombinatorialNullstellensatz}
		Let $\mathbb{F}$ be a field and let $f \in \mathbb{F}[x_1,x_2,\ldots,x_n]$ be a polynomial such that $\deg(f) = \sum_{i=1}^n t_i$ and the coefficient of $\prod_{i=1}^n x_i^{t_i}$ is non-zero.  Let $S_1,S_2,\ldots,S_n$ be subsets of $\mathbb{F}$ such that $|S_i| > t_i$ for all $i \in [n]$.  Then there exists $(s_1,s_2,\ldots,s_n) \in S_1 \times S_2 \times \cdots \times S_n$ such that $f(s_1,s_2,\ldots,s_n) \neq 0$.
	\end{lem}
	
	\begin{prop}[Variation of Theorem 6.1 in \cite{alon}]\label{Prop-SubgraphNullstellensatzResult}
		Let $p$ be a prime number in $[\tfrac{n}{4},\tfrac{n}{2}]$.  Then $\mathcal{M}(G)$ contains a subgraph $U$ such that for every $u \in V(U)$, $\deg(u) \in \{p,2p, 3p,4p,5p,6p,7p\} \cap [2(n-1)]$.  %In particular, $U$ contains vertices in each part of $\mathcal{M}$, each with degree at least $p$.
	\end{prop}
	\begin{proof}
		We define a polynomial $f$ with degree $|E(\mathcal{M})|$ over $\mathbb{F}_2$, and using the fact that $a^{p-1}$ $\Mod{p}$ $\equiv 1$ for all $a \not\equiv 0 \Mod{p}$, we show the existence of the desired subgraph using the nullstellensatz directly.
		
		Define the polynomial
		$$f(x_e:e \in E(\mathcal{M})) = \prod_{v \in V(\mathcal{M})} \biggr[1 - \Big(\sum_{\substack{e \in E(\mathcal{M})\\ v \in e}}x_e \Big)^{p-1} \biggr] - \prod_{e \in E(\mathcal{M})}(1-x_e).$$
		
		The degree of $f$ is $|E(\mathcal{M})|$ because
		$$|V(\mathcal{M})| (p-1) = (2n-1)(p-1) \leq (2n-1)(\tfrac{n}{2}-1) = n(n-\tfrac{5}{2})+1 < n(n-1) = |E(\mathcal{M})|.$$
		
		Note that the max degree term, $(-1)^{|E(\mathcal{M})|}\prod_{e \in E(\mathcal{M})} x_e$ has a non-zero coefficient. 
		To apply the nullstellensatz, we consider solutions to $f$ of the form $(s_1,s_2,\ldots,s_{|E(\mathcal{M})|}) \in \{0,1\}^{|E(\mathcal{M})|}$ (where $t_i = 1$ for all $i \in [|E(\mathcal{M})|]$).  Thus by Lemma \ref{Lemma-CombinatorialNullstellensatz}, there exists a vector, call it $r=(r_e:e \in E(\mathcal{M}))$, such that $f(r) \neq 0$.  By the definition of $f$, $r \neq 0$ because $f(0) = 0$, so some of its entries are $1$.  This means that the latter product in $f$ vanishes when evaluated at $r$.  The former product in $f$ can be non-zero only when $\Big(\sum_{\substack{e \in E(\mathcal{M})\\ v \in e}}r_e \Big)^{p-1} \equiv 0 \Mod{p}$.  It follows that $r$ corresponds to a subgraph $U$ of $\mathcal{M}(G)$ whose vertex degrees are congruent to $0 \Mod{p}$.  Since $\Delta(\mathcal{M}) = 2(n-1)$ and $r \neq 0$, there exists a vertex $u \in U$ such that $\deg(u) \in \{p,2p, 3p, 4p, 5p, 6p, 7p\} \cap [2(n-1)]$.  Note that since the degrees of the vertices in the neighbourhood of $u$ are all at least $1$, $U$ contains at least one vertex in each part of $\mathcal{M}$ with degree at least $p$. \qedhere
	\end{proof}

	%\begin{cor}
	%	Let $L(G)$ be a crescent labelling of a graph $G$ and $p$ be a prime number in $[\tfrac{n}{4}, \tfrac{n}{2}]$.  Then there exists a vertex $v \in V(G)$ and a distance $d \in \{d_1,d_2,\ldots,d_{n-1}\}$ such that there are at least $p$ other vertices in $G$ at distance $d$ from $v$.
	%\end{cor}
	
	\section{Distance Multiplicities in Unweighted Graphs}
	
	Let $G$ be a tree. Define $T(G)$ to be $G$ without its leaves, and on each vertex $v$ of $T(G)$ assign it a weight equal to the degree of $v$ in $G$, $\deg(v)$.  Define $m(k)$ to be the multiplicity of distance $k$ in a graph $G$.
	
	The following expresses $m(k)$ in terms of the degrees of the vertices of $G$, or equivalently, the vertex weights in $T(G)$.
	
	\begin{lem}[Characterizing Distance Multiplicities in Terms of Vertex Degrees]\label{Lemma-CharacterizingMultiplicityInTermsOfDegrees}
		It holds that $m(1) = |E(G)|$, $m(2) = \sum_{v \in V(G)} {\deg(v) \choose 2}$, and when $3 \leq k \leq \diam(G)$, 
		$$m(k) = \sum_{\substack{\{x,y\} \subset T(G) \\ d(x,y) = k-2}} (\deg(x)-1)(\deg(y)-1).$$
	\end{lem}

	\begin{proof}[Proof sketch]
		The cases $k = 1$ and $k=2$ are straightforward and no distance can be larger than the diameter of $G$.  Suppose $3 \leq k \leq \diam(G)$.  Let $x,y \in T(G)$ where $d(x,y) = k-2$ and let $P(x,y)$ be the unique path of length $k-2$ between $x$ and $y$.  There are $\deg(x)-1$ and $\deg(y)-1$ neighbours of $x$ and $y$ in $G$ that are not in $P(x,y)$.  Let $w$ be such a neighbour of $x$ and $z$ such a neighbour of $y$.  Then the unique $(w,z)$-path contains $P(x,y)$ and has length $k$.  Thus $d(w,z) = k$ and there are $(\deg(x)-1)(\deg(y)-1)$ such pairs.  So, each pair $x,y \in T(G)$ satisfying $d(x,y) = k-2$ contributes a multiplicity for $k$ in $G$ of $(\deg(x)-1)(\deg(y)-1)$.  
		
		It is because $G$ is a tree that this method counts all instances of distance $k$; if $G$ has a cycle, then some distances can be over counted and this sum is an upper bound for $m(k)$. \qedhere
	\end{proof}
	
	\subsection{Conjectures}
	
	\begin{conj}
		Let $d$ be the largest distance that attains maximum multiplicity in a tree $T$.  Then for every $i \in \{d,\ldots, \diam(T)-1\}$, $m(i) \geq m(i+1)$.
	\end{conj}

	\begin{rem}
		I suspect it is possible to prove this by induction on the path lengths $k$.  That is, every path of length $k+1$ corresponds to at least $1$ distinct path of length $k$.  But I think things get a bit tricky because somehow the maximality of $d$ needs to come into play.
	\end{rem}

	\begin{rem}
		\textbf{There are counter-examples to the related claim} that $m(i) \geq m(i-1)$ for all $i \in \{d,\cdots,2\}$.
	\end{rem}

	The remaining conjectures are all about upper bounding $m(d)$.  The following proposition handles the lower bound.
	
	\begin{prop}
		Let $d$ be the largest distance with max multiplicity in a tree $T$.  If $1 \leq d \leq \lceil \frac{n}{3} \rceil$, then $m(d) \geq n-1$.
	\end{prop}

	\begin{proof}
		It is sufficient to construct a tree $T$ such that $m(d) = |E(T)| = n-1$.  Let $u$ be a root vertex.  Append two paths $X$ and $Y$ of length $d-1$ to $u$. Then for the remaining $n-2(d-1)-1$ vertices, append them as a length $n-2(d-1)-1$ path to $u$.  There are $3(d-1)$ distinct paths of length $d$ with endpoints in $X \cup Y$.  There are $n-2(d-1)-d$ paths of length $d$ with endpoints in $V(T) \setminus (X \cup Y)$. 
		
		Altogether, there are $n-2d+2-d + 3d-3 = n-1$ paths of length $d$ in $T$.  Note that since $d \leq \ceil{n/3}$, 
		$$n-2(d-1)-d \geq n-3\ceil{n/3}+2 = \begin{cases}
			0, \text{ if } n \equiv 1 \Mod{3}	\\
			1, \text{ if } n \equiv 2 \Mod{3},
		\end{cases}$$ and in either case, $m(d) = n-1$.  Observe that in fact $m(1) = m(2) = \cdots = m(d) = n-1$. \qedhere
	\end{proof}
	
	\begin{eg}
		Figure \ref{Figure-ExtremalTreeExampleMinimizingMultiplicity} shows a tree with maximum multiplicity $m(d) = n-1$ where $d=6$ is the largest distance with max multiplicity and $n = 20$.
		\begin{figure}[h]
			\centering
			\includegraphics[scale=0.65]{ExtremalTreeExample3(min).png}
			\caption{Extremal tree example minimizing $m(d)$.}\label{Figure-ExtremalTreeExampleMinimizingMultiplicity}
		\end{figure}
	\end{eg}

	\begin{rem}
		Below I conjecture that $d \leq \ceil{n/3}+2$.  I have not yet looked for extremal trees that minimize $m(d)$ when $d \in \{\ceil{n/3}+1, \ceil{n/3}+2\}$.
	\end{rem}
	\newpage
	\begin{conj}
		Let $d$ be the largest distance with max multiplicity in a tree $T$.  
		\begin{enumerate}
			\item If $d \leq C_1\frac{n}{3} + C_2$ and even, then $m(d) \leq (3-a-b)\lceil \frac{r}{3} \rceil \lfloor \frac{r}{3} \rfloor + \lfloor \frac{r}{3} \rfloor^{2a} \lceil \frac{r}{3} \rceil^{2b}$, where $r = n-\tfrac{3}{2}d+2$ and 
			$$(a,b) = \begin{cases}
				(1,0), \text{ if } r \equiv 1 \Mod{3}	\\
				(0,1), \text{ if } r \equiv 2 \Mod{3}	\\
				(0,0), \text{ otherwise. }
			\end{cases}$$
			
			\item If $C_1\frac{n}{3} +C_2< d \leq \lceil \tfrac{n}{3} \rceil + 2$, then $m(d) \leq a\lfloor \frac{r'}{4} \rfloor^2 + (2-a)\lceil \frac{r'}{4} \rceil^2 + 2\lfloor\frac{r'}{4} \rfloor \lceil\frac{r'}{4} \rceil$, where $r' = n-d-1$ and 
			$$a = \begin{cases}
				2, \text{ if } r' \equiv 1 \Mod{4}	\\
				1, \text{ if } r' \equiv 2 \Mod{4}	\\
				0, \text{ otherwise. }
			\end{cases}$$
		\end{enumerate}
	\end{conj}
	
	\begin{rem}
		My experimentation suggests that $C_1 \sim 1$ and $-1 \leq C_2 \leq 1$; however, I have not yet examined these values carefully.
	\end{rem}
	
	
	\begin{rem}
		I believe there are $3$ extremal trees that maximize $m(d)$; one unknown when $d \leq C_1\frac{n}{3} + C_2$ and odd, and the other two are described below.  
	\end{rem}
	
	\paragraph{Construction 1:} Refer to Figure \ref{Figure-ExtremalTreeExamples-a} for an example.  When $d \leq C_1\frac{n}{3}+C_2$ and even, do the following:
	\begin{enumerate}
		\item First we use $3(\frac{d}{2} - 1) + 1$ vertices by making $3$ branch paths with length $\frac{d}{2}-1$ from a root vertex $u$.  
		
		\item Let $v,w,x$ be the vertices at the ends of each branch.  
		
		\item For the remaining $n-3(\frac{d}{2} - 1) - 1$ vertices, append them to $v,w,x$ so that the number of leaf neighbours of $v$, $w$, and $x$ differ from one another by at most $1$.
	\end{enumerate}

	\begin{rem}
		Trees with large $m(d)$ when $d  \leq C_1\frac{n}{3} + C_2$ often tend to have a triple branching structure.  The structure of $T$ becomes much more constrained the larger $d$ gets, and I think this is probably because it is most common for $d = 2$.  When $d > 2$, then for $m(2) \leq m(d)$ to hold, \textbf{(1)} the degrees of the vertices of $T$ cannot be too high, and \textbf{(2)} there needs to be enough branching in $T$ to ensure enough distinct length $d$ paths.  Somehow the triple branching pattern in Construction 1 satisfies \textbf{(1)} and \textbf{(2)} while also maximizing $m(d)$; but I doubt that this extremal structure is fragile.  That is, I think even when $d$ is odd and $d  \leq C_1\frac{n}{3} + C_2$, an extremal tree has a similar triple branching structure.
	\end{rem}
	

	\paragraph{Construction 2:} Refer to Figure \ref{Figure-ExtremalTreeExamples-b} for an example.  When $d > C_1\frac{n}{3} + C_2$, do the following:
	\begin{enumerate}
		\item Form a path of length $d-4$ and call its leaves $x$ and $y$.
		
		\item Append two vertices $x_1$ and $x_2$ to $x$ and similarly $y_1$ and $y_2$ to $y$.
		
		\item Append the remaining $n-d-1$ vertices to $x_1$, $y_1$, $x_2$, and $y_2$ so that the number of leaf neighbours on each differ from one another by at most $1$.  If $r' \equiv 2 \Mod{4}$, then ensure that both $x_1$ and $y_1$ are each adjacent to $\ceil{r'/4}$ leaves.
	\end{enumerate}

	\begin{eg}
		Figure \ref{Figure-ExtremalTreeExamples} shows examples from Constructions 1 and 2, which are mentioned above.  In Figure \ref{Figure-ExtremalTreeExamples-a}, $n = 20$, $d = 6$, and $m(6) = 56$.  In Figure \ref{Figure-ExtremalTreeExamples-b}, $n = 24$, $d=\lceil \frac{n}{3} \rceil + 2 = 10$, and $m(d) = 42$.
		
		\begin{figure}[h]
			\centering
			\begin{subfigure}{6.5cm}
				\includegraphics[scale=0.65]{ExtremalTreeExample2.png}
				\caption{Construction 1}\label{Figure-ExtremalTreeExamples-a}
			\end{subfigure}
			\begin{subfigure}{6.5cm}
				\includegraphics[scale=0.65]{ExtremalTreeExample1.png}
				\caption{Construction 2.}\label{Figure-ExtremalTreeExamples-b}
			\end{subfigure}
			\caption{Extremal tree examples that maximize $m(d)$.}\label{Figure-ExtremalTreeExamples}
		\end{figure}
	\end{eg}

	\begin{conj}
		Let $d$ be the largest distance with max multiplicity.  Then $d \leq \lceil \tfrac{n}{3}\rceil + 2$.
	\end{conj}

	\begin{rem}
		I have not yet found a counter-example to this conjecture.  Please let me know if you find one!  I have searched $n \leq 25$ without finding a CE, but it may well be that $d \leq \ceil{n/3} + C\sqrt{n}$ or something.  If so, then there would probably still be a sensible case division at $d \sim n/3$.
	\end{rem}
	
	
	\iffalse
	\subsection{Constructions}
	
	\begin{prop}
		Let $d$ be the largest distance with max multiplicity.  Then $\min_{T \in \mathcal{T}(n)}m(d) = n-1$.  Let $r = n-3(d/2-1)-1$ and $r' = n-d+1$ and $d$ even.  Then 
		$$\max_{T \in \mathcal{T}(n)}m(d) = \begin{cases}
			\lfloor \tfrac{r}{3} \rfloor(\lfloor \tfrac{r}{3} \rfloor + 2\lceil\tfrac{r}{3} \rceil) & \text{ if } d \leq \tfrac{n}{3}, \text{ and } r \equiv 0,1 \Mod{3}; 	\\
			\lceil \tfrac{r}{3} \rceil(\lceil \tfrac{r}{3} \rceil + 2\lfloor\tfrac{r}{3} \rfloor) & \text{ if } d \leq \tfrac{n}{3}, \text{ and } r \equiv 2 \Mod{3}; 	\\
			
			2\lfloor\tfrac{r'}{4}\rfloor^2 + 2\lfloor\tfrac{r'}{4}\rfloor \lceil\tfrac{r'}{4}\rceil, & \text{ if } d > n/3 \text{ and } r' \equiv 0,1 \Mod{4};	\\
			\lfloor\tfrac{r'}{4}\rfloor^2 + 2\lfloor\tfrac{r'}{4}\rfloor \lceil\tfrac{r'}{4}\rceil + \lceil\tfrac{r'}{4}\rceil^2, & \text{ if } d > n/3 \text{ and } r' \equiv 2 \Mod{4};	\\
			2\lceil\tfrac{r'}{4}\rceil^2 +2\lfloor\tfrac{r'}{4}\rfloor \lceil\tfrac{r'}{4}\rceil, & \text{ if } d > n/3 \text{ and } r' \equiv 3 \Mod{4}.
		\end{cases}$$
	\end{prop}

	\begin{proof}
		When $d \leq \tfrac{n}{3}$, we construct a tree $T$ in two steps.  First we use $3(d/2 + 1) + 1$ vertices by making $3$ branches with length $d-1$ from a root vertex $u$.  Let $v,w,x$ be the vertices at the ends of each branch, respectively.  For the remaining $n-3(d/2 + 1) + 1$ vertices, append them to $v,w,x$ so that the degrees of $v$, $w$, and $x$ differ from one another by at most $1$.  Note that since $d \leq \tfrac{n}{3}$, so $m(2) \leq m(d)$.
	\end{proof}
	\fi
	
	\section{Crescent Vertices}
	
	Let $v$ be a vertex of a graph $G$.  We say that $v$ is a \emph{crescent vertex} if the multiset of distances from $v$ to every other vertex in $G$ is of the form $\{d_1^{1},d_2^2,\ldots,d_k^{k}\}$ for some $k$.  For example, every vertex in the $4$-cycle $C_4$ is a crescent vertex.  A crescent vertex $v$ has the property that the rest of the vertices can be partitioned into $k$ classes based on the distance from $v$ such that the numbers of vertices in each class is given by a permutation.  We call such a permutation a \emph{crescent permutation}.  For instance, each crescent vertex in $C_4$ induces the crescent permutation $(2,1)$, or $(12)$.  Note that a vertex $v$ is crescent if and only if every other vertex similar to $v$ is crescent, so we are concerned with finding orbits of the given graph that contain crescent vertices.
	
	\newpage
	\begin{thebibliography}{100}
		\bibitem{alon} Alon, N. (1999). Combinatorial Nullstellensatz. Combinatorics, Probability and Computing, 8(1-2), 7-29. doi:10.1017/S0963548398003411
	\end{thebibliography}
	
\end{document}