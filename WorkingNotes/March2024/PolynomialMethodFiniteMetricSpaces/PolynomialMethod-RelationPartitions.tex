\documentclass[12pt]{article}

\usepackage{geometry}
\usepackage{amsmath, amsthm, amssymb}
\usepackage{graphicx}
\usepackage{tikz}
\usepackage{tkz-berge}
\usepackage{tkz-graph}
\usepackage{booktabs} % See the package documentation for guidelines on formal tables: https://ctan.org/pkg/booktabs
\usepackage{verbatim} % Used to typeset, for example, code snippets or pseudo-code for algorithms.
\usepackage{dsfont} % Extra fontset for helpful mathematics symbols, e.g. \mathds{1}
\usepackage{etoolbox} % Used to allow boolean variables for use in the title page
\usepackage{import}
\usepackage{lipsum}
\usepackage{subcaption}
\usepackage{float}
\usepackage{enumitem}
\usepackage{tabularx}
\usepackage{array}
\usepackage{pdfpages}
\usepackage{mathtools}
\usepackage{hyperref}
\usepackage{mathbbol}
\usepackage{caption}

\newcolumntype{C}[1]{>{\centering\arraybackslash}m{#1}}
\newcommand{\R}{\mathbb{R}}
\newcommand{\Q}{\mathbb{Q}}
\newcommand{\C}{\mathbb{C}}
\newcommand{\N}{\mathbb{N}}
\newcommand{\Z}{\mathbb{Z}}
\newcommand{\T}{\mathbb{T}}
\newcommand{\cA}{\mathcal{A}}
\newcommand{\cB}{\mathcal{B}}
\newcommand{\cD}{\mathcal{D}}
\newcommand{\cP}{\mathcal{P}}
\newcommand{\cM}{\mathcal{M}}
\newcommand{\abs}[1]{\left\lvert #1 \right\rvert}
\newcommand{\norm}[1]{\left\lVert #1 \right\rVert}
\newcommand{\set}[2]{\left\{#1 \ : \ #2\right\}}
\newcommand{\conv}[1]{\underset{#1}\longrightarrow}
\newcommand{\Mod}[1]{\ (\mathrm{mod}\ #1)}
\newcommand{\Supp}[0]{\ \mathrm{Supp}\ }
\DeclarePairedDelimiter\ceil{\lceil}{\rceil}
\DeclarePairedDelimiter\floor{\lfloor}{\rfloor}
\DeclareMathOperator{\lcm}{lcm}

\newcommand{\Cross}{\mathbin{\tikz [x=1.4ex,y=1.4ex,line width=.2ex] \draw (0,0) -- (1,1) (0,1) -- (1,0);}}

\newcommand\restr[2]{{% we make the whole thing an ordinary symbol
		\left.\kern-\nulldelimiterspace % automatically resize the bar with \right
		#1 % the function
		\vphantom{\big|} % pretend it's a little taller at normal size
		\right|_{#2} % this is the delimiter
}}
% Custom math operators (analogous to \lim, \sup, etc).
\DeclareMathOperator{\id}{id}
\DeclareMathOperator{\od}{od}
\DeclareMathOperator{\subspan}{span}
\DeclareMathOperator{\sgn}{sgn}
\DeclareMathOperator{\diam}{diam}
\DeclareMathOperator{\rad}{rad}
%\DeclareMathOperator{\span}{span}

\newtheorem{thm}{Theorem}[section] % Numbering is impacted by [chapter]; could do [section] or [subsection] also.
\newtheorem{lem}{Lemma} % The [thm] argument says to number Lemma in sequence with Theorem.
\newtheorem{prop}[thm]{Proposition}
\newtheorem{cor}[thm]{Corollary}
\newtheorem{conj}[thm]{Conjecture}
\newtheorem{question}{Question}
% These environments are unnumbered and will not count toward the numbering.
%\newtheorem*{question}{Question}
\newtheorem*{answer}{Answer}
\newtheorem*{conjecture}{Conjecture}
\newtheorem*{claim}{Claim}
% These environments are definitions; they have a different style (bold label, standard font).
\theoremstyle{definition}
\newtheorem{defn}[thm]{Definition} % These definitions are also numbered in sequence with Theorem.
\newtheorem{eg}{Example}
\newtheorem{rem}[thm]{Remark}
\newtheorem{obs}{Observation}

\title{ \vspace{-3cm} General Necessary Conditions on Relation Partitions Using the Combinatorial Nullstellensatz}
\author{Tao Gaede}

\begin{document}
	\maketitle
	
	\section{Summary}
	The basic structure of interest here is a finite set $X$ paired with a partition of its unordered pairs into symmetric relations $R_1, \ldots, R_r$.  We denote this partition with $P$, and we call the pair $(X,P)$ a \emph{relation partition}.  First, we present Lemma \ref{Lem-NullstellensatzEigenvalueInterlacing}, which is a general cardinality condition on a subset of $\{R_1, \ldots, R_r\}$ in term of positive multiples of large primes and laplacian eigenvalue partial sums of graphs corresponding to the $R_k$s.  We prove this lemma using the combinatorial nullstellensatz and eigenvalue interlacing.  Using Lemma \ref{Lem-NullstellensatzEigenvalueInterlacing}, we briefly discuss implications on the structure of crescent configurations in $\R^2$ of order $9$.  Then we move on to a multiple set variation, wherein we have a family of relation partitions $(X_1,P_1), \ldots, (X_s,P_s)$ satisfying $P_i \subseteq\{R_1, \ldots, R_r\}$.  We obtain Theorem \ref{Thm-MainMultipleSetResult} using Lemma \ref{Lem-NullstellensatzEigenvalueInterlacing}, which provides general necessary conditions on both the structure and cardinalities of $R_1, \ldots, R_r$.  We discuss a technique for how Theorem \ref{Thm-MainMultipleSetResult} could be used to derive stronger conditions in more particular contexts.  We include a brief discussion on entropy as a natural way to describe the complexity of a relation partition.
	
	\iffalse
	\section{Preliminaries}
	
	Bipartite Distance Multigraph, Distance Graphs, Distance Multiplicities as adj mat (or Laplacian) traces,
	
	Combinatorial Nullstellensatz, Eigenvalue interlacing theorem
	
	\section{Nullstellensatz Shows Existence of Dense Subgraph}
	
	\section{General Position (Max Distance Degree) Permits Large Order}
	
	\begin{thm}
		Let $X$ be a finite metric space with $n$ points and $r$ distinct distances $d_1, \ldots, d_r$ whereby either $n/2$ or $(n+1)/2$ is a prime $p$.  Let $d \in \Z^+$.  Suppose $\Delta(D_{d_k}) \leq d$ for all $k \in [1,r]$.  Then there is a set of $P$ points and $D$ distances satisfying $|P| \geq |D| \geq p/d$ such that 
		\begin{enumerate}
			\item for each $v \in P$, there are $p$ points $u \in P \setminus \{v\}$ such that $d(u,v) \in D$; and
			\item for each $d_k \in D$, there is an $\ell_k \in \Z^+$ such that there are $\ell_k p$ points $P$ at distance $d_k$ with some other point in $X$.
		\end{enumerate}
	\end{thm}

	\begin{cor}
		The average multiplicity of distances in $D$ is at least $\frac{|P|p}{2|D|}$.
	\end{cor}
	
	\section{Eigenvalue Interlacing and Distance Multiplicity Stability}
	Let $A_k$ be the adjacency matrix for distance $d_k$ in $X$.  Then $2m(d_k) = Tr(A_k^2) = \deg_{\mathcal{M}}(d_k)$.  We say that a point $v \in X$ is \emph{distance independent} if for all $u,w \in X \setminus \{v\}$, $d_{X \setminus \{v\}}(u,w) = d_{X}(u,w)$.
	
	Let $v$ be a distance independent vertex of $X$.  Let $\alpha_1^{(k)} \geq \alpha_2^{(k)} \geq \cdots \geq \alpha_n^{(k)}$, and $\beta_1^{(k)} \geq \beta_2^{(k)} \geq \cdots \geq \beta_{n-1}^{(k)}$ be the eigenvalues of $A_k(X)$ and $A_k(X \setminus \{v\})$, respectively.  Similarly, let $\lambda_1^{(k)} \geq \lambda_2^{(k)} \geq \cdots \geq \lambda_n^{(k)}$, and $\mu_1^{(k)} \geq \mu_2^{(k)} \geq \cdots \geq \mu_{n-1}^{(k)}$ be the eigenvalues of $\mathcal{L}(A_k(X))$ and $\mathcal{L}(A_k(X \setminus \{v\}))$, respectively.
	\begin{thm}
		  Let $X$ and $Y$ finite metric spaces that differ by a single distance independent point.  Then for every $k \in [1,r]$, 
		$$|m_X(k) - m_{Y}(k)| < \tfrac{1}{2}(\alpha_1^2 + \alpha_n^2),$$
		and
		$$|m_X(k) - m_{Y}(k)| \leq \tfrac{\lambda_1}{2}.$$
	\end{thm}
	\begin{proof}
		Without loss of generality, suppose $Y = X \setminus \{v\}$, where $v$ is a distance independent point.  For ease of notation, we omit the $(k)$ superscripts in the eigenvalues for this proof.  
		
		For the Laplacians $\mathcal{L}(A_k(X))$ and $\mathcal{L}(A_k(Y))$, we have by eigenvalue interlacing that $\lambda_1 \geq \mu_1 \geq \cdots \geq \mu_{n-1} \geq \lambda_n$, which immediately implies
		$$Tr\mathcal{L}(A_k(X)) - \lambda_1 = \sum_{j=2}^n \lambda_j \leq \sum_{j=1}^{n-1}\mu_j \leq \sum_{j = 1}^{n-1}\lambda_j = Tr\mathcal{L}(A_k(X)) - \lambda_n.$$
		Since $Tr\mathcal{L}(A_k(Y)) = \sum_{j=1}^{n-1}\mu_j$, we have that $|m_X(k) - m_Y(k)| \leq \tfrac{1}{2}(\lambda_1 - \lambda_n)$.  But the laplacian is singular with all eigenvalues non-negative, so $\lambda_n = 0$.
		
		For the adjacency squares, since $v$ is independent, by eigenvalue interlacing, we again have
		$$\alpha_1 \geq \beta_1 \geq \alpha_2 \geq \beta_2 \geq \cdots \geq \alpha_{n-1} \geq \beta_{n-1} \geq \alpha_n.$$
		Let $i$ be the smallest integer satisfying $\alpha_i< 0$.  Then for all $j \in [i,n-1]$, $\alpha_j^2 \leq \beta_j^2 \leq \alpha_{j+1}^2$, and similarly, for all $j \in [1,i-2]$, $\alpha_j^2 \geq \beta_j^2 \geq \alpha_{j+1}^2$.  Note that if $i = 2$, then $\beta_1$ and $\alpha_1$ are the only positive eigenvalue of $A_k(Y)$ and $A_k(X)$, respectively, which means that $\beta_1 = \sum_{j=2}^{n-1}|\beta_j| \geq \sum_{j=2}^{n-1}|\alpha_j| \geq |\alpha_2|$.  Thus we have that
		$$\sum_{j=1}^n\alpha_j^2 - \alpha_1^2 - \alpha_n^2 \leq \sum_{j = 1}^{n-1}\beta_j^2 = \sum_{j=1}^{i-1} \beta_j^2 + \sum_{j=i}^{n-1}\beta_j^2 \leq \sum_{j=1}^{n}\alpha_j^2 - \alpha_i^2.$$
		Thus the multiplicity gap for distance $d_k$ between $X$ and $Y$ is 
		$$|m_X(k) - m_Y(k)| \leq \frac{1}{2}(\alpha_1^2 + \alpha_n^2 - \alpha_i^2) < \frac{1}{2}(\alpha_1^2 + \alpha_n^2).$$
		Note that $\alpha_i$ is one of the eigenvalues of $A_k(X)$ with smallest magnitude.
	\end{proof}

	\begin{thm}
		Let $X$ be a finite metric space with $n$ points and $r$ distinct distances $d_1, d_2, \ldots, d_r$.  Let $p$ be a prime satisfying $p-1 < \tfrac{n(n-1)}{n+r}$.  Let $\mathcal{U} = (P,D)$ be a subgraph of $\mathcal{M}(X)$ such that for all $v \in P$, $\deg_{\mathcal{U}}(v) = h_vp$ with $h_v \in \Z^+$ and for all $d_k \in D$, $\deg_{\mathcal{U}}(d_k) = \ell_k p $ with $\ell_k \in \Z^+$.  Then for all $d_k \in D$, it holds that
		$$\sum_{j=n-|P|+1}^n \lambda_j^{(k)} \leq \ell_k p.$$
	\end{thm}

	\fi
	\section{Relation Partition and Their Cardinalities}
	Let $X$ be a finite set of $n$ elements, called points.  Let $R_1, \ldots, R_r$ be symmetric relations that partition the unordered pairs of points in $X$.  We call such a partition a \emph{relation partition of $X$}.  Then for each $k \in [r]$, we define $G_k$ to be the graph corresponding to $R_k$; that is, $G_k$ has vertex set $X$ where $u\sim v$ if and only if $\{u,v\} \in R_k$.  
	
	We are interested in the set $\{|R_k|: k \in [r]\}$.  Observe that $|R_k| = \tfrac{1}{2}Tr(L(G_k))$, where $L$ denotes the Laplacian matrix.  We will use the combinatorial nullstellensatz (Lemma \ref{Lem-CombinatorialNullstellensatz}) and eigenvalue interlacing.
	
	\begin{lem}[Combinatorial Nullstellensatz]\label{Lem-CombinatorialNullstellensatz}
		Let $\mathbb{F}$ be a field and let $f \in \mathbb{F}[x_1,x_2,\ldots,x_n]$ be a polynomial such that $\deg(f) = \sum_{i=1}^n t_i$ and the coefficient of $\prod_{i=1}^n x_i^{t_i}$ is non-zero.  Let $S_1,S_2,\ldots,S_n$ be subsets of $\mathbb{F}$ such that $|S_i| > t_i$ for all $i \in [n]$.  Then there exists $(s_1,s_2,\ldots,s_n) \in S_1 \times S_2 \times \cdots \times S_n$ such that $f(s_1,s_2,\ldots,s_n) \neq 0$.
	\end{lem}

	%\begin{lem}[Eigenvalue Interlacing]\label{Lem-EigenvalueInterlacing}
	%	...
	%\end{lem}
	
	We say that a relation partition $R_1,\ldots, R_r$ of $X$ is \emph{dependent} if there exists a subset $S \subseteq X$ such that for some $v \in S$ there is an $x \in X$ satisfying $R_k(v,x),$ and $u,w \in X \setminus S$ satisfying $R_{k'}(u,w)$ such that $R_k(v,x) \Leftrightarrow R_{k'}(u,w)$.  That is, deleting the vertices of $S$ causes point pairs outside of $S$ to no longer be related.  For example, if $(X; R_1,\ldots, R_r)$ is a metric space for a graph in which $R_k$ corresponds to the distinct distance $k$, any vertex cut destroys geodetic paths between vertex pairs in the different resulting components.  Note that dependence occurs here because the geodetic paths are in $X$ itself rather than an underlying space.  If on the other hand, $X$ is a subset of an underlying metric space $\mathcal{X}$ whereby the distances are calculated with respect to geodetic paths in $\mathcal{X}$, then no $(X \subseteq \mathcal{X};R_1,\ldots, R_r)$ can be dependent.  We say that $(X: R_1,\ldots, R_r)$ is \emph{independent} if it is not dependent.
	
	\begin{lem}\label{Lem-NullstellensatzEigenvalueInterlacing}
		Let $X$ be a finite set of $n$ points with an independent relation partition $R_1, \ldots, R_r$.  Then for any prime $p$ satisfying $p-1 < \tfrac{n(n-1)}{n+r}$, there exist nonempty subsets $D \subseteq [r]$ and $P \subseteq X$, such that for all $k \in D$, there is a positive integer $\ell_k$ such that $\ell_k p \leq 2|R_k|$, and if $X\setminus P$ is independent, $\ell_k p \leq \sum_{j =1}^{|P|}\lambda_j^{(k)}$,
		%$$\ell_k p \leq \sum_{j =1}^{|P|}\lambda_j^{(k)},$$
		where $\lambda_j^{(k)}$ is the $j$-th largest eigenvalue of $L(G_k)$.
	\end{lem}
	\begin{proof}[Proof of Lemma \ref{Lem-NullstellensatzEigenvalueInterlacing}]
		Let $\mathcal{M}$ be the bipartite multigraph with point vertices $X$ in one part and relation vertices for each $R_k, k\in [r]$ for the other part.  For each $k \in [r]$, and for each $u,v \in X$ satisfying $R_k(u,v)$ we include the edges $u\sim R_k$ and $v \sim R_k$.
		\begin{claim}[Variation of Theorem 6.1 in \cite{alon}]\label{Prop-SubgraphNullstellensatzResult}
				The multigraph $\mathcal{M}(X)$ contains a non-empty subgraph $\mathcal{U}$ such that for every $u \in V(\mathcal{U})$, $\deg(u) \in \{kp: k \in \Z^+\}$.  %In particular, $U$ contains vertices in each part of $\mathcal{M}$, each with degree at least $p$.
		\end{claim}
			\begin{proof}[Proof of Claim]
				We define a polynomial $f$ with degree $|E(\mathcal{M})|$ over $\mathbb{F}_2$, and using the fact that $a^{p-1}$ $\Mod{p}$ $\equiv 1$ for all $a \not\equiv 0 \Mod{p}$, we show the existence of the desired subgraph using the nullstellensatz directly.
				Define the polynomial
				$$f(x_e:e \in E(\mathcal{M})) := \prod_{v \in V(\mathcal{M})} \biggr[1 - \Big(\sum_{\substack{e \in E(\mathcal{M})\\ v \in e}}x_e \Big)^{p-1} \biggr] - \prod_{e \in E(\mathcal{M})}(1-x_e).$$
				The degree of $f$ is $|E(\mathcal{M})|$ because every other term has degree at most
				$$|V(\mathcal{M})| (p-1) = (n+r)(p-1) < n(n-1) = |E(\mathcal{M})|.$$
				Note that the max degree term of $f$, $(-1)^{|E(\mathcal{M})|+1}\prod_{e \in E(\mathcal{M})} x_e$, has a non-zero coefficient. 
				To apply the nullstellensatz, we consider solutions to $f$ of the form $(s_1,$ $s_2,$ $\ldots,$ $s_{|E(\mathcal{M})|}) \in \{0,1\}^{|E(\mathcal{M})|}$ (where $t_i = 1$ for all $i \in [|E(\mathcal{M})|]$).  Thus by Lemma \ref{Lem-CombinatorialNullstellensatz}, there exists a edge vector $\mathbf{u}=(u_e:e \in E(\mathcal{M}))$ such that $f(\mathbf{u}) \neq 0$.  By the definition of $f$, $\mathbf{u} \neq \mathbf{0}$ because $f(\mathbf{0}) = 0$, so some of its entries are $1$.  This means that the latter product in $f$ vanishes when evaluated at $\mathbf{u}$.  The former product in $f$ can be non-zero only when
				$$\Big(\sum_{\substack{e \in E(\mathcal{M})\\ v \in e}}u_e \Big)^{p-1} \equiv 0 \Mod{p}.$$  
				It follows that $\mathbf{u}$ corresponds to a subgraph $\mathcal{U}$ of $\mathcal{M}(X)$ whose vertex degrees are congruent to $0 \Mod{p}$.  Since $\mathbf{u} \neq \mathbf{0}$, there exists a vertex $v \in \mathcal{U}$ such that $\deg_{\mathcal{U}}(v) \in \{kp: k \in \Z^+\}$.  Note that since $\mathcal{U}$ is a subgraph of $\mathcal{M}$, which is bipartite, the degree sums in each part need to be equal; therefore, the vertices of $\mathcal{U}$ all have degrees being a positive multiple of $p$, and these positive degree vertices are in both parts. \qedhere
			\end{proof}
		Notice that the claim implies that for each $k \in D$, there exists a positive integer $\ell_k$ such that $\deg_{\mathcal{U}}(R_k) = \ell_k p$, which means in general $2|R_k| \geq \ell_k p$.  
		
		Suppose $X \setminus P$ is independent.  Our goal is to lower bound the degrees of a subset of the relation vertices of $\mathcal{M}$.  Using what we know about the degrees of $\{R_k: k \in D\}$ from the subgraph $\mathcal{U}$ we obtained in the claim above, we now apply eigenvalue interlacing on the laplacians of the $G_k$ graphs for $k \in D$ on the complement of the point vertices of $\mathcal{U}$.
		
		Let $Y$ be the point vertices of $\mathcal{U}$, and set $m := |Y|$.  For each $k \in D$, let $\mu_{1}^{(k)} \geq \mu_2^{(k)} \geq \cdots \geq \mu_{n-m}^{(k)}$ be the laplacian eigenvalues of $G_k \setminus Y$.  Then by eigenvalue interlacing, we have that for each $j \in [n-m]$, $\lambda_{m+j}^{(k)} \leq \mu_j^{(k)}$.  For each $v \in Y$, let $N_k(v)$ be the $k$-neighbourhood of $v$ in $X$; that is, $N_k(v)$ is the set of other points $u \in X$ satisfying $R_k(v,u)$.  Note that $\deg_{G_k}(v) = |N_k(v)|$.  Since $\sum_{j=1}^{n-m} \mu_j^{(k)}$ counts the degree of $R_k$ in $\mathcal{M}$ involving edges incident to point vertices only in $X \setminus Y$ (we don't lose count of the crossing edges here because the diagonal entries of the principal submatrix of the Laplacian are unchanged), it follows that
		$$\sum_{j=1}^n \lambda_{j}^{(k)} = \sum_{j=1}^{n-m} \mu_j^{(k)} + \sum_{v \in Y} \deg_{G_k}(v).$$
		Since $\ell_k p = \deg_{\mathcal{U}}(R_k) \leq \sum_{v\in Y} \deg_{G_k}(v)$, it follows that for each $k \in D$,
		$$\sum_{j=m+1}^n \lambda_j^{(k)} \leq \sum_{j=1}^n \lambda_j^{(k)} - \ell_kp \Leftrightarrow \ell_k p \leq \sum_{j=1}^m \lambda_j^{(k)},$$
		as desired.
	\end{proof}
	
	
	
	The value that the nullstellensatz approach in Lemma \ref{Lem-NullstellensatzEigenvalueInterlacing} is providing is that we get a set of relations of larger cardinality, rather than only one relation.  Note that PHP implies that there is some relation $R_k$ with cardinality at least $\ceil{\tfrac{n(n-1)}{4r}}$. Lemma \ref{Lem-NullstellensatzEigenvalueInterlacing} gives us a \textbf{set} $D$ of relations with ``large" cardinality, and we can control the size of $D$ to ensure that we have a larger set of relations with this cardinality.  What follows are conditions that allow us to control $|Y|$ and $|D|$.  This will help us prove necessary structure on relation partitions in particular contexts (see Example \ref{Example-CrescentsOrder9} for a brief discussion on implications for crescents in $\R^2$ of order $9$).  %In particular, we prove a necessary lower bound on the maximum relation cardinality in various contexts.  Note that in metric spaces, this is the same thing as lower bounding the maximum multiplicity of a distinct distance.
	
	There is a special case when $n/2$ or $(n+1)/2$ is prime since in this case the point vertices of the subgraph of $\mathcal{M}$ given by Lemma \ref{Lem-NullstellensatzEigenvalueInterlacing} must have degree $p$.
	
	\begin{cor}
		If $\ceil{n/2}$ is prime, then $|Y| \geq |D|$ and 
		$$|Y| p \leq \sum_{k \in D}\sum_{j =1}^{|Y|}\lambda_j^{(k)}.$$
	\end{cor}
	\begin{proof}
		Since no point can be related with more than $n-1$ other points, the point vertices in the multi-subgraph $\mathcal{U}$ from the proof of Lemma \ref{Lem-NullstellensatzEigenvalueInterlacing} have degree exactly $p = \ceil{n/2}$.  Additionally, since $\mathcal{U}$ is bipartite, the degree sums in both the point and relation parts need to be equal, so $|Y| = \sum_{k \in D}\ell_k$. \qedhere
	\end{proof}

	The next corollary introduces a natural condition that seems to show up in a variety of relation partition contexts, including in the context of geometric finite metric spaces where the condition of general position with respect to spheres is considered.  For example, suppose $X$ is a finite metric space in $\R^{d-1}$ such that we forbid any $d+1$ points being on a $(d-2)$-sphere $\mathcal{S}^{d-2}$; then, this implies that no point in $X$ can be at equal distance with more than $d$ other points in $X$.
	\begin{cor}
		If for each $k \in [r]$, it holds that $\Delta(G_k) \leq d$ for some positive integer $d$, then $|D| \geq \ceil{p/d}$.
	\end{cor}
	\begin{proof}
		Each point vertex $v$ in $Y$ has degree at least $p$, and there are at most $d$ edges between $v$ and any relation vertex in $D$. \qedhere
	\end{proof}
	
	The purpose of the following example is to show how Lemma \ref{Lem-NullstellensatzEigenvalueInterlacing} can imply necessary substructure in $X$, if $X$ and its relation partition are instantiated with particular properties (that is, as a metric space, or proper edge colouring, an equivalence relation, or whatever).
	
	\begin{eg}[Crescent Configurations of Order $9$ in the plane]\label{Example-CrescentsOrder9}
		Let $X$ be a crescent configuration in general position with respect to the circle in $\R^2$ with $9$ points.  Let $d_1,\ldots, d_8$ be the distinct distances of $X$.  Note that we are in the case $p = 5$ and $d = 3$.  Since distances are calculated in the underlying space of $\R^2$, all subsets of $X$ are independent.  Thus by Lemma \ref{Lem-NullstellensatzEigenvalueInterlacing} there exist non-empty subsets $D \subseteq [8]$ and $Y \subseteq X$ with $|Y| \geq |D|$ such that for all $k \in D$, there exists a positive integer $\ell_k$ satisfying $5\ell_k \leq \sum_{j=1}^{|Y|}\lambda_j^{(k)}$, where $\lambda_j^{(k)}$ is the $j$-th largest laplacian eigenvalue of the distance graph $G_k$ corresponding to distance $d_k$.  Since the maximum multiplicity of each distance is $9-1 = 8$, the distance graphs are either trees or disconnected.
		
		Notice that $p = 5$ and the max multiplicity of $8$ condition implies $1 \leq |D| \leq 6$ (there is a distance in $D$ with multiplicity at least $\ceil{p/2} = 3$).  Indeed $|D| \geq 2$, since each point vertex in $Y$ has degree exactly $p = 5$ and $d = 3$.  In the case $|Y| = |D| = 2$, we have a particular ``distance saturated" configuration occurring in $X$ whereby two points $x_1,x_2 \in Y$ and two distances $d_1,d_2 \in D$ satisfy the properties that $x_1$ is at distance $d_1$ and $d_2$ with three and two other points, respectively, and similarly $x_2$ is at distance $d_2$ and $d_2$ with three and two other points, respectively.  Note that if $d(x_1,x_2) \notin \{d_1,d_2\}$, then the multiplicity of both $d_1$ and $d_2$ is at least $5$.  Also, if $|Y| > |D|$, then there exists a distance $d_k$ with $\ell_k \geq 2$, giving it multiplicity at least $5$; we also have that $\ell_k \leq 3$ and at most one distance can satisfy $\ell_k = 3$ since $\ceil{3p/2} = 8$, which is the max multiplicity.  There are probably other implications of Lemma \ref{Lem-NullstellensatzEigenvalueInterlacing}.
		
		It could be fruitful to examine the possible distance graphs of order $9$ crescent configurations and see if one could narrow down the possible candidates using the structure of $Y$ and $D$ with the laplacian eigenvalue necessary condition.  There seems to be a fair amount of structure to work with here.
	\end{eg}
	
	\iffalse
	Let $X_1,\ldots, X_s$ be finite sets with cardinalities $n_1,\ldots, n_s$.  Suppose these sets have independent relation partitions $\{X_i: R_1^{i},\ldots, R_r^{i}\}$, where some $R_k^{i}$ may be empty.  For each $k \in [r]$ set $R_k = \cup_{i=1}^s R_k^i$.  Set $n:= |\bigcup_{i \in [s]}X_i|$.  Suppose $\ceil{n/2}$ is prime.  There are $\sum_{i \in [s]}{n_i \choose 2}$ unordered pairs and $r$ relations.  Then if $\ceil{n/2} -1 < \frac{\sum_{i \in [s]}{n_i \choose 2}}{n+r}$, we get a nice subgraph and can do some good stuff!....
	
	\begin{thm}
		Let $X_1,\ldots, X_s$ be finite sets with cardinalities $n_1,\ldots, n_s$ satisfying $n_i \in [N-\delta, N]$ with $\ceil{(N-\delta)/2}$ prime.  Suppose these sets have independent relation partitions $\{X_i: R_1^{i},\ldots, R_r^{i}\}$, where some $R_k^{i}$ may be empty.  For each $k \in [r]$ set $R_k = \cup_{i=1}^s R_k^i$.  Suppose for each $k \in [r]$, $\Delta(G_k) \leq d$.  Then there exists a set $I \subset [s]$ satisfying $|I| \leq \frac{\sqrt{s}}{2d}$ such that
		$$\biggr|\bigcup_{i \in I}X_i \biggr| \geq \frac{\sqrt{s}\ceil{(N-\delta)/2}}{2d^2}.$$
	\end{thm}
	
	Let $X_1, \ldots, X_s$ be finite sets with cardinalities $n_1,\ldots, n_s$, respectively.  Let $R_1, \ldots, R_r$ be a relation partition of the union of unordered pairs from each of $X_1, \ldots, X_s$.  Let $p$ be the largest prime satisfying $p-1 < \tfrac{n_i(n_i-1)}{n_i+r}$ for all $i \in [s]$.  Then there exists a $D \subseteq [r]$ and $m \geq 1$ such that for all $k \in D$, there exists a nonnegative integer sequence $(\ell_k^{(1)}, \ldots, \ell_k^{(s)})$, not all $0$ such that
	$$ \sum_{i = 1}^s\ell_k^{(i)}p \leq \sum_{j=1}^m \lambda_j^{(k)}.$$
	
	[proof of something like this]
	
	Under what conditions can we ensure that all subgraphs given by the nullstellensatz share a relation vertex?  Such conditions provide a natural lower bound on a maximum cardinality of a relation.  The next theorem presents such a condition.
	
	\begin{thm}
		Suppose there exists an integer $t$ satisfying $t(t-1) = \sum_{i=1}^s n_i(n_i-1)$.  Suppose each $G_k$ has max degree $d$, $r = t-1$, and $s \geq Cd^2$... other conditions needed.  Then there exists a $k \in [r]$ such that $|R_k| \geq t$.
	\end{thm}

	\begin{rem}
		Notice that the above theorem provides a general necessary condition for the existence of a crescent family.
	\end{rem}
	\fi

	\begin{rem}[Laplacian partial sums]
		There are known general upper bounds on partial sums of Laplacian eigenvalues, which could be used if needed.  There's a conjectured general upper bound of $\sum_{j=1}^m\lambda_j \leq (\#\text{ of edges}) + {m+1 \choose 2}$, which is known to hold for trees and all graphs of order at most $10$.
	\end{rem}
	\section{Multiple Set Version}
	
	Let $X_1,\ldots, X_s$ be finite sets with symmetric relation partitions $P_1, \ldots, P_s \subseteq \{R_1, \ldots, R_r\}$.  Observe that $R_1,\ldots, R_r$ are symmetric relations that partition $\bigcup_{i \in [s]}{X_i \choose 2}$.  Our goal is to understand the cardinalities of $R_1,\ldots, R_r$.  In Theorem \ref{Thm-MainMultipleSetResult} below, we apply Lemma \ref{Lem-NullstellensatzEigenvalueInterlacing} to each constituent (point set, partition) pair $(X_i,P_i)$ to obtain ``dense" relation sets $D_i$ for each $i \in [s]$.  Then using these sets with the max degree $d$ condition on each set, we determine that there must exist some positive integer $t$ satisfying four necessary conditions on the structure of $R_1, \ldots, R_r$.  The idea for how to use Theorem \ref{Thm-MainMultipleSetResult} in a particular context is to determine the permissible values of $t$ in this context, which then in turn provides a set of necessary conditions on the relation cardinalities (ideally these are quite restricted conditions given the particular context).
	
	Recall we say that a set $X_i$ has \emph{max degree $d$} in $P_i$ if for each $k \in [r]$ and $v \in X$, there are at most $d$ points $u\in X \setminus \{v\}$ such that $\{u,v\} \in R_k$.  Again, this condition is equivalent to the max degree of the graph $G_k$ being at most $d$.
	
	\begin{thm}\label{Thm-MainMultipleSetResult}
		Let $s$, $r$, and $d$ be positive integers and $X_1,\ldots, X_s$ be finite sets with max degree $d$ and symmetric relation partitions $P_1,\ldots,P_s$, respectively, where each $P_i \subseteq \{R_1, \ldots, R_r\}$.  Let $p$ be the largest prime satisfying $p \leq \min_{i \in [s]}\big(\tfrac{|X_i|(|X_i|-1)}{|X_i|+|P_i|} \big)$.  Then there exists a positive integer $t$ and an integer $\ell \in [0, r-\tfrac{sp^2}{2td}]$ such that the following conditions hold:
		\begin{enumerate}
			\item the inequality $r \geq \tfrac{sp^2}{2td}$ holds;
			\item there is some $k \in [r]$ such that $|R_k| > t$;
			\item there are $\tfrac{sp^2}{2td} + \ell$ relations with cardinality at least $p/2$; and
			\item there is a set of relations $\{R_{j_q}: q \in [\tfrac{sp^2}{2td}]\}$ such that 
			$$|R_{j_q}| \geq \biggr(\frac{\tfrac{sp}{d}-(q-1)2t/p}{\tfrac{sp^2}{2td} + \ell-(q-1)} \biggr)\frac{p}{2}.$$
		\end{enumerate}
		%if $t \leq \tfrac{sp^2}{2rd}$,
		%If every collection $\mathcal{D} = \{D_i \subseteq P_i: |D_i| \geq \ceil{p/d}, i \in [s]\}$ with size exactly $\floor{2t/p}$ satisfies $\bigcap_{D \in \mathcal{D}}D \neq \emptyset$, then 
		%there exists a $k \in [r]$ such that $|R_k|>t$.
		%It holds that if for every $I \subset [s]$ satisfying $|I| \leq 2r/p$, every collection of the form $\{Y_i\subseteq X_i: i \in I, |Y_i| \geq \tfrac{p}{3d}\}$ satisfies $|\bigcup_{i \in I}Y_i| > \tfrac{2rN}{p}$, then there exists $k \in [r]$ such that $|R_k| > r$.
	\end{thm}

	\begin{proof}
		Let $\{(Y_i,D_i): i \in [s]\}$ be the bipartite multigraphs obtained from Lemma \ref{Lem-NullstellensatzEigenvalueInterlacing} applied to each $(X_i,P_i)$ with the prime $p$.  We have by max degree that each relation vertex in $(Y_i,D_i)$ has at most $d$ edges with each point vertex, so $|D_i| \geq \ceil{p/d}$.  Suppose $t$ is maximal satisfying the property that there is a subcollection $\mathcal{D}$ of $\{D_1,\ldots, D_s\}$ of size at least $2t/p$ such that $\bigcap_{D \in \mathcal{D}}D \neq \emptyset$.  Then there exists a $k \in [r]$ satisfying $|R_k| > t$.  By maximality of $t$, each relation $R_1, \ldots, R_r$ is contained in fewer than $2t/p$ of the sets $D_1,\ldots, D_s$.  Consider the $(0,1)$ incidence matrix $A$ with rows corresponding to the relations in $\bigcup_{i \in [s]}D_i$ and columns the sets $D_1,\ldots, D_s$.  The row sums of $A$ are at most $2t/p$ and the column sums are at least $p/d$, and since there are $s$ columns, we need enough rows $r'$ to satisfy $r' (2t/p) \geq sp/d$.  Thus $r \geq |\bigcup_{i \in [s]}D_i| \geq \tfrac{sp^2}{2td}$, and recall that all relations in $\bigcup_{i\in[s]}D_i$ have cardinality at least $p/2$.  Suppose $r' = \tfrac{sp^2}{2td} + \ell$, and note that all of these rows are non-zero.  Rearrange the rows of $A$ by row sum from highest to lowest.  The average number of $1$s in these $r'$ rows is $\tfrac{sp}{dr'}$, so there exists a row $r_1$ with this many $1$s; suppose $r_1$ has max row sum.  Delete $r_1$ from $A$ and what remains is a matrix with at most $sp/d - 2t/p$ $1$s and $r'-1$ rows.  By induction on $q$, for all $q \in [r'+\ell]$, there is a row $r_q$ with at least $(\tfrac{sp}{d}-(q-1)2t/p)\tfrac{1}{(r'-(q-1))}$ $1$s.  Therefore, there exists a set of distinct relations $\{R_{j_q}: q \in [r']\}$ such that $|R_{j_q}| \geq (\tfrac{sp}{d}-(q-1)2t/p)\tfrac{p}{2(r'-(q-1))}$. \qedhere
		
		
		
		%Therefore, there exists at least $\tfrac{sp^2}{2td}$ relations $R_1,\ldots, R_r$ with cardinality at least $p/2$. \qedhere
		
		%We have $s$ sets of relation vertices of size at least $\ceil{p/d}$, so by pigeonhole principle, there is a $J \subseteq [s]$ satisfying $|J| \geq \ceil{\tfrac{sp}{rd}}$ such that $\bigcap_{i \in J}D_i \neq \emptyset$. %By assumption, every $I \in {[s] \choose \floor{2t/p}}$ satisfies $\bigcap_{i \in I}D_i \neq \emptyset$, 
		%Thus there is some relation vertex $R_k$ in this intersection with degree at least $\tfrac{sp}{rd}p$.  Choose $s_0$ such that $\tfrac{s_0p}{rd}p = 2t \Leftrightarrow s_0 = \tfrac{2trd}{p^2}$.  So, for $s > s_0$, $R_k$ has degree greater than $2t$ in $\bigcup_{i \in [s]}(Y_i,D_i)$, implying that $|R_k| > t$. \qedhere 
		%Our goal is to upper bound $|\bigcup_{i \in I}Y_i|$ using what we know about $\{D_i: i \in I\}$. Since $X_i$ has max degree $d$, so does $Y_i$, thus each point vertex in $(Y_i,D_i)$ has at least $\ceil{p/d} \geq p/d$ relation vertex neighbours, which means that $|D_i| \geq p/d$. Since $p$ is chosen to be largest satisfying $p \leq \ceil{|X_i|/2}$ and each $|X_i| \geq N/2$ with $\ceil{N/4}$ prime, we have $p \geq N/4$; this means that no point vertex in $(Y_i,D_i)$ can have degree greater than $3p$, and so $|Y_i| \geq \tfrac{|D_i|}{3} \geq \tfrac{p}{3d}$.  Thus 
		%$$\biggr|\bigcup_{i \in I}Y_i \biggr| \leq |I||Y_i| \leq \frac{2r}{p} N.$$
	\end{proof}
	
	Very little is assumed about $R_1,\ldots, R_r$ or $P_1,\ldots, P_s$, other than that the relation graphs, when restricted to $(X_i,P_i)$ have max degree $d$.  That is, we haven't specified any cardinality lists for the relations in each $P_i$.  The format of the theorem suggests a technique: Given a relation partition of a family of point sets $X_1, \ldots, X_s$, find the values of $t$ that satisfy the conditions of Theorem \ref{Thm-MainMultipleSetResult}; then, these permissible $t$ values constrain the cardinalities of the relations.  Since Theorem \ref{Thm-MainMultipleSetResult} is so general, the idea would be to constrain to relation partitions of set families with specific structure and then find permissible $t$ values, which imply necessary conditions on the relation cardinalities by Theorem \ref{Thm-MainMultipleSetResult}.  Below are some examples of what I mean by ``specific structure".  What are the permissible values of $t$ if 
	\begin{enumerate}
		\item each point set is an AP in $\Z$ or $\Z_n$ (each relation graph is a forest -- and I think they tend to be paths? So, $d \leq 2$);
		\item each $(X_i,P_i)$ is an optimal proper edge colouring of $K_{|X_i|}$ using a subset $P_i$ of $r$ colours (note proper implies $d=1$ here);
		\item each $(X_i,P_i)$ is a metric space of points in $\R^{d-1}$ that are in general position with respect to the sphere $\mathcal{S}^{d-2}$ (so for $\R^2$, $d=3$); or
		\item what if each $(X_i,P_i)$ is a coherent configuration?  (I don't know anything about this, but it seems quite related, and I'd be interested to learn more!)
	\end{enumerate}
	
	The max degree parameter $d$ seems to come up in a variety of contexts, and it's often just a small constant.  This suggests to me that the max degree condition is a natural one.  Also, it's neat how there's this general interplay between the structure of the relation graphs $G_k$ (max degree and the laplacian), the degrees of the bipartite multigraph $\mathcal{M}$ (nullstellensatz) and the actual combinatorics of the underlying relation partitions (metric spaces, edge colourings, and so on).
	%But nonetheless, we can still lower bound many of the relation cardinalities.  The format of the theorem suggests a technique for finding necessary conditions on cardinality lists: suppose the maximum cardinality is less than $t$, then there must be $\tfrac{sp^2}{2td} + \ell$ relations with cardinality at least $p/2$ and if $\ell$ is small, many of these relations have even larger cardinality.  The simplest application follows:
	
	For all this to work, we needed a ``dense subgraph existence" argument like Lemma \ref{Lem-NullstellensatzEigenvalueInterlacing} to get us a \textbf{subset} of large cardinality relations for each pair $(X_i,P_i)$.  I think Lemma \ref{Lem-NullstellensatzEigenvalueInterlacing} is a proof of concept of this idea, and it would be interesting to explore other techniques or conditions that enable us to find dense subgraphs of $\mathcal{M}$ to make a similar but stronger version of Theorem \ref{Thm-MainMultipleSetResult} work.
	
	\section{Entropy Connection to Relation Cardinalities}
	
	The normalized relation cardinalities can be interpreted as a probability vector (each value is essentially the counting measure of one of the relations).  Let $p_1, \ldots, p_r$ be these probabilities for relations $R_1,\ldots,R_r$.  So, for each $k \in [r]$, the probability that a randomly chosen unordered pair is in $R_k$ is $p_k$.  Then the entropy of this probability vector is $-\sum_{k=1}^r p_k\log(p_k)$, which by Jensen's inequality is at most $\log(r)$ with equality in the case when all relation cardinalities are equal.  The idea is that the entropy of the relation partition can be interpreted as a statistic describing the complexity of the partition, and this complexity depends on the relation cardinalities.  Suppose one has found necessary relation cardinality bounds on the relation partitions that satisfy specific structure, like being particular class of metric space, or a coherent configuration, or anything.  Then these cardinality bounds imply entropy bounds, which means one can in a sense ``bound the complexity" of a class of structures that are defined in terms of relation partitions.  
	
	For example, this perspective actually provides a neat motivation for characterizing cases when uniform relation cardinalities occur -- a while back, I noticed a little proof that trees can't have uniform distance multiplicities, except for $K_2$ and $K_{1,3}$, which means that trees as metric spaces have entropy strictly less than $\log(\diam)$.  Odd cycles have uniform distance multiplicities, so they have entropy $\log(\diam)$.  Crescents (cardinalities being exactly $1, \ldots, r$) have entropy $-\sum_{k = 1}^r \frac{k}{{r+1\choose 2}} \log(\tfrac{2k}{r(r+1)})$, which I think is close, but not equal to $\log(r)$.
	
	Inversely, suppose we assume that the entropy of some class of relation partitions is $g$.  What does this tell us about the relation cardinalities?  It might be possible to use the statistic of entropy as a way to compare relation partitions.
	
	\section{Conclusion}
	I think Lemma \ref{Lem-NullstellensatzEigenvalueInterlacing} and Theorem \ref{Thm-MainMultipleSetResult} provide a decent proof of concept for a set of tools, derived from techniques (like the nullstellensatz) for finding dense subgraphs of the bipartite multigraph $\mathcal{M}$.  Such tools enable one to find necessary conditions on the structure and cardinalities of relation partitions, both for single sets and set families.  These results could probably be extended to non-symmetric relations and also to $b$-ary relation partitions wherein we partition the $b$-subsets (or $b$-tuples).  There appears to be a rather deep connection between the graph structure in the relation graphs of a relation partition, the combinatorics of particular classes of relation partitions, and the structure of the bipartite multigraph $\mathcal{M}$.
	
	%\section{Extending This to Compact Metric Spaces}
	
	%Suppose we have a relation partition $R_1, \ldots, R_r$ of some finite metric space $X$ with order $n$ that is contained in an underlying space $\mathcal{X}$.  The pairs in $X \times X = \{x_{i,j}: 1 \leq i,j \leq n\}$ are a finite subset of $\mathcal{X} \times \mathcal{X}$.  Let $\mathcal{C}$ be a subset of $\mathcal{X} \times \mathcal{X}$ that contains $X \times X$ but ignores the finitely many points in $\mathcal{X} \times \mathcal{X}$ that are at equal distance to at least two points in $X \times X$ (since there are finitely many, we're ignoring a set of measure 0).  Now define a function $f: \mathcal{C} \rightarrow \{R_1, \ldots, R_r\}$ that maps points $y \in \mathcal{X} \times \mathcal{X}$ to the nearest point in $X \times X$.  Suppose $(\mathcal{C},\mu)$ is a measure space.  Define a new measure $\nu$ on $\mathcal{C}$ as follows: For $A \subset \mathcal{C}$, we partition $A$ into subsets $A_1,\ldots,A_r$ and set $\nu(A) := \sum_{k = 1}^r \mu(A_k)\tfrac{|R_k|}{{n \choose 2}}$.  For example, suppose $X \subset \R^2$.  Then we can think of $\mathcal{X} \times \mathcal{X}$ as $\C^2$ or $\R^4$ and use Lebesgue measure for $\mu$.
	
	%As mentioned in the entropy section above, we can think of the normalized relation cardinalities as a probability measure.
	
	%Note that crescents have entropy
	%\begin{align*}
	%	-\sum_{k = 1}^r \frac{k}{{r+1\choose 2}} \log(\tfrac{2k}{r(r+1)}) &\sim -\frac{1}{{r+1\choose 2}}\int_{1}^r x\log(\tfrac{2x}{r(r+1)})dx	\\
	%	&= \biggr[ \frac{1}{4}x^2(2\log(\tfrac{2x}{r(r+1)})-1) \biggr]\biggr|_1^r	\\
	%	&= \biggr[ \frac{1}{4}r^2(2\log(\tfrac{2r}{r(r+1)})-1) \biggr] - \biggr[ \frac{1}{4}(2\log(\tfrac{2}{r(r+1)})-1) \biggr]	\\
	%	&= \biggr[ \frac{1}{4}r^2(2\log(\tfrac{2}{r+1})-1) \biggr] - \biggr[ \frac{1}{4}(2\log(\tfrac{2}{r(r+1)})-1) \biggr]
		%&= -\frac{1}{{r+1\choose 2}}\sum_{k = 1}^r k (\log(k) - \log{r+1 \choose 2})	\\
		%&= -\frac{1}{{r+1\choose 2}}\sum_{k = 1}^r k \log(k) + \frac{\log{r+1 \choose 2}}{{r+1\choose 2}}\sum_{k = 1}^rk	\\
		%&=-\frac{1}{{r+1\choose 2}}\sum_{k = 1}^r k \log(2k) + \frac{1}{{r+1\choose 2}}\sum_{k = 1}^r k \log(r(r+1))	\\
		%&=-\frac{1}{{r+1\choose 2}}\sum_{k = 1}^r k \log(2k) + \log(r) + \log(r+1)	\\
		%&= -\frac{\log(2)}{{r+1\choose 2}}\sum_{k = 1}^r k -\frac{1}{{r+1\choose 2}}\sum_{k = 1}^r k \log(k) + \log(r) + \log(r+1)	\\
		%&= -\log(2) -\frac{1}{{r+1\choose 2}}\sum_{k = 1}^r k \log(k) + \log(r) + \log(r+1)	\\
		%&= -\frac{1}{{r+1\choose 2}}\sum_{k = 1}^r k \log(k) + \log{r+1 \choose 2}	\\
		%&= -\frac{1}{{r+1\choose 2}}\log \biggr(\prod_{k = 1}^r k^k\biggr)+ \log{r+1 \choose 2}	\\
	%\end{align*}
	 
	
	%Notice that the condition in Theorem \ref{Thm-MainMultipleSetResult} cannot be met if $\tfrac{2trd}{p^2} \geq s$.  While I currently don't see why one would have $d$ grow as a function of $s$, there are cases when both $t$ and $r$ grow with $s$.  For example in crescent families of sets $X_i$ with max degree $d$ (note the constituent sets are not assumed to be crescents here), if we wanted to contradict this structure by showing that there is some $t > r$, we would run into trouble because the binomial condition requires $r(r-1)= \sum_{i\in[s]}|X_i|(|X_i|-1)$.  If we assume the balanced case (to maximize $p$, say) where we assume $|X_i| \sim N$ and so at best $p \sim N/2$, we would have roughly that $r \sim \sqrt{s}N$, and since we want $t \geq r$, we have that $s \leq \tfrac{2trd}{p^2} \leq 8sd$, which means the condition in Theorem $\ref{Thm-MainMultipleSetResult}$ cannot be met.  Although, I think the condition is close to being met here; so I suspect there's a way to somehow improve Theorem \ref{Thm-MainMultipleSetResult} (and possibly add a condition) to get some kind of upper bound on the number of sets in a crescent family.
	
	%The following is a necessary lower bound on the maximum relation cardinality for set families with constituent sets having max degree $d$.
	
	%\begin{cor}
	%	If $X_1,\ldots,X_s$ are finite sets with symmetric relation partitions $P_1,\ldots, P_s \subseteq \{R_1,\ldots, R_r\}$ with max degree $d$.  Let $p$ be the largest prime satisfying $p \leq \min_{i \in [s]}\big(\tfrac{|X_i|(|X_i|-1)}{|X_i|+|P_i|} \big)$.  Then $\max_{k\in[r]}(|R_k|) > \tfrac{sp^2}{2rd}$.  
	%\end{cor}
	%\begin{proof}
	%	We contradict Theorem \ref{Thm-MainMultipleSetResult} otherwise. 
	%\end{proof}
	
	%It is probably a good idea to think of Theorem \ref{Thm-MainMultipleSetResult} as a max relation cardinality (max multiplicity) sufficient condition.  It is a sufficient condition to ensure the max cardinality is at least $t$ (supposing we are allowed to add sufficiently many sets with max degree $d$).
	
	%\begin{cor}
	%	Let $s$, $r$, and $d$ be positive integers and $X_1,\ldots, X_s$ be finite sets with max degree $d$ and symmetric relation partitions $P_1,\ldots,P_s$, respectively, where each $P_i \subseteq \{R_1, \ldots, R_r\}$.  Let $p$ be the largest prime satisfying $p \leq \max_{i \in [s]}(\ceil{|X_i|/2})$ and let $t \in [\tfrac{p}{2}, \tfrac{pr}{2}]$.  If $\bigcap_{i\in [s]}P_i$ has cardinality at least $t- \ceil{p/d}+1$, then there exists a $k \in [r]$ such that $|R_k|>t$.
	%\end{cor}

	%\begin{cor}
	%	Let $X_1,\ldots,X_s$ be finite metric spaces satisfying $\sum_{i\in[s]}{|X_i| \choose 2} = {t+1 \choose 2}$ for some $t \in \N$.  Suppose the internal distinct distances of these sets are in $\{d_1,\ldots,d_{t}\}$.  Suppose each point in $X_i$ is at equal distance to at most $d$ other points in $X_i$.  Let $p$ be the largest prime satisfying $p \leq \max_{i \in [s]}(\ceil{|X_i|/2})$.  Let $P_1,\ldots, P_s$ be the distance sets of $X_1,\ldots, X_s$, respectively, where each $P_i \subseteq \{d_1,\ldots,d_{t-1}\}$.  Suppose $\bigcap_{i\in[s]}P_i$ has cardinality at least $t - \ceil{p/d}+1$.  Then there exists a distance with multiplicity greater than $t$.
	%\end{cor}
	
	\begin{thebibliography}{100}
		\bibitem{alon} Alon, N. (1999). Combinatorial Nullstellensatz. Combinatorics, Probability and Computing, 8(1-2), 7-29. doi:10.1017/S0963548398003411
	\end{thebibliography}
	
\end{document}