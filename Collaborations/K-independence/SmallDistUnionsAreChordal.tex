\documentclass[12pt]{article}

\usepackage{geometry}
\usepackage{amsmath, amsthm, amssymb}
\usepackage{graphicx}
\usepackage{tikz}
\usepackage{booktabs} % See the package documentation for guidelines on formal tables: https://ctan.org/pkg/booktabs
\usepackage{verbatim} % Used to typeset, for example, code snippets or pseudo-code for algorithms.
\usepackage{dsfont} % Extra fontset for helpful mathematics symbols, e.g. \mathds{1}
\usepackage{etoolbox} % Used to allow boolean variables for use in the title page
\usepackage{import}
\usepackage{lipsum}
\usepackage{subcaption}
\usepackage{float}
\usepackage{enumitem}
\usepackage{tabularx}
\usepackage{array}
\usepackage{pdfpages}
\usepackage{mathtools}
\usepackage{hyperref}
\newcolumntype{C}[1]{>{\centering\arraybackslash}m{#1}}
\newcommand{\R}{\mathbb{R}}
\newcommand{\Q}{\mathbb{Q}}
\newcommand{\C}{\mathbb{C}}
\newcommand{\N}{\mathbb{N}}
\newcommand{\Z}{\mathbb{Z}}
\newcommand{\T}{\mathbb{T}}
\newcommand{\cA}{\mathcal{A}}
\newcommand{\cB}{\mathcal{B}}
\newcommand{\cD}{\mathcal{D}}
\newcommand{\cP}{\mathcal{P}}
\newcommand{\cM}{\mathcal{M}}
\newcommand{\abs}[1]{\left\lvert #1 \right\rvert}
\newcommand{\norm}[1]{\left\lVert #1 \right\rVert}
\newcommand{\set}[2]{\left\{#1 \ : \ #2\right\}}
\newcommand{\conv}[1]{\underset{#1}\longrightarrow}
\newcommand{\Mod}[1]{\ (\mathrm{mod}\ #1)}
\newcommand{\Supp}[0]{\ \mathrm{Supp}\ }
\DeclarePairedDelimiter\ceil{\lceil}{\rceil}
\DeclarePairedDelimiter\floor{\lfloor}{\rfloor}
\DeclareMathOperator{\lcm}{lcm}

\newcommand{\Cross}{\mathbin{\tikz [x=1.4ex,y=1.4ex,line width=.2ex] \draw (0,0) -- (1,1) (0,1) -- (1,0);}}

\newcommand\restr[2]{{% we make the whole thing an ordinary symbol
		\left.\kern-\nulldelimiterspace % automatically resize the bar with \right
		#1 % the function
		\vphantom{\big|} % pretend it's a little taller at normal size
		\right|_{#2} % this is the delimiter
}}
% Custom math operators (analogous to \lim, \sup, etc).
\DeclareMathOperator{\id}{id}
\DeclareMathOperator{\od}{od}
\DeclareMathOperator{\subspan}{span}
\DeclareMathOperator{\sgn}{sgn}
\DeclareMathOperator{\diam}{diam}
\DeclareMathOperator{\rad}{rad}

\newtheorem{thm}{Theorem}[section] % Numbering is impacted by [chapter]; could do [section] or [subsection] also.
\newtheorem{lem}{Lemma} % The [thm] argument says to number Lemma in sequence with Theorem.
\newtheorem{prop}[thm]{Proposition}
\newtheorem{cor}[thm]{Corollary}
\newtheorem{conj}[thm]{Conjecture}
\newtheorem{question}{Question}
% These environments are unnumbered and will not count toward the numbering.
%\newtheorem*{question}{Question}
\newtheorem*{answer}{Answer}
\newtheorem*{conjecture}{Conjecture}
\newtheorem*{claim}{Claim}
% These environments are definitions; they have a different style (bold label, standard font).
\theoremstyle{definition}
\newtheorem{defn}[thm]{Definition} % These definitions are also numbered in sequence with Theorem.
\newtheorem{eg}{Example}
\newtheorem{rem}[thm]{Remark}
\newtheorem{obs}{Observation}

\title{ \vspace{-3cm} Chordal Distance Graphs }
%\author{Tao Gaede}

\begin{document}
	\maketitle
	
	Let $T$ be a tree and $d:V(T) \times V(T) \rightarrow \N_0$ the standard graph metric.  Define $D_{\leq k}:= D_{\leq k}(T)$ to be the graph with vertex set $V(T)$ and for each $u,v \in V(T)$, $uv$ is an edge if and only if $d(u,v) \leq k$ in $T$.  We similarly define $D_{=k}$ to be the $k$-distance graph.  We show below that $D_{\leq k}$ is chordal.
	
	\section{Small Distance Graphs of Trees are Chordal.}
	\begin{lem}\label{Lemma-IntersectingVertex}
		Let $T$ be a tree containing paths $P = (p_0,$ $p_1,$ $\ldots,$ $p_{k-1})$ and $Q = (q_0,$ $q_1,$ $\ldots,$ $q_{k-1})$ such that $p_{k-1} = q_0$ and $p_0 \neq q_{k-1}$.  Then there is a unique vertex $u$ satisfying $u \in P \cap Q$, and $N(u) \cap (P \triangle Q) \neq \emptyset$.  That is, $u$ intersects both $P$ and $Q$ and has a neighbour only in $P$ and a neighbour only in $Q$.
	\end{lem}
	\begin{proof}
		Since $T$ is a tree and $P$ and $Q$ have a unique common end vertex, the set of common vertices between $P$ and $Q$ must induce a path ending at this end vertex, otherwise $P$ and $Q$ would induce a cycle.\qedhere
	\end{proof}
	
	\begin{lem}\label{Lemma-WheelCase}
		Suppose $D_{\leq k}$ has an induced cycle $C = (v_0,v_1, \ldots, v_{\ell-1},v_0)$.  For each $i \in [0,\ell-2]$ define $P_i$ to be the unique $(v_i,v_{i+1})$-path in $T$.  Define $u_{i,j}$ to be the unique intersecting vertex between paths $P_i$ and $P_j$ that has a neighbour only in $P_i$ and a neighbour only in $P_j$.  Then the subgraph of $T$ induced by $\{P_i: i \in [0,\ell-2]\}$ contains a spinal path with vertex subsequence $(v_0,u_{0,1}, u_{1,2}, \ldots, u_{\ell-3,\ell-2}, v_{\ell - 1})$, where $v_0$ and $v_{\ell-1}$ are the ends of the spine.  Specifically, $P_0$ only intersects $P_1$ and $P_{\ell-2}$ only intersects $P_{\ell-3}$, and for each $i \in [1, \ell-3]$, $P_i$ intersects exactly $P_{i-1}$ and $P_{i+1}$.
	\end{lem}

	\begin{proof}
		%Suppose $P_0$ intersects with $P_2$.  Let $u_{0,2}$ be the intersecting vertex of $P_0$ and $P_2$.  Since $d(v_0,v_1) = d(v_1,v_2)$, $P_0$ and $P_1$ must intersect at some vertex $u_{0,1}$.  
		
		%Let $d(v_1,u_{0,1}) = a_1$.  Then $d(v_0,u_{0,1}) = d(u_{0,1}, v_2) = k-a_1$.  Note that since $d(v_2,v_3) = k$, $d(u_{0,2},v_3) = a_1$, which means that if $d(v_0,u_{0,2}) > d(v_0,u_{0,1})$, $k < d(v_1,v_3) < 2a_1$.  But then $d(v_0,v_2) = 2k - 2a_1 < 2k - k = k$, which means $C$ has a chord, a contradiction.  So, $d(v_0,u_{0,2}) \leq d(v_0,u_{0,1})$.  However, in this case, since $d(v_2,v_3) = k$, $d(u_{0,2},v_3) \leq a_1$, but again this means that $k < d(v_1,v_3) \leq 2a_1$, which forces a chord between $v_0$ and $v_2$ as above.
		
		Since $T$ is a tree, for each $i \in [0,\ell-2]$, $P_i$ is the unique $(v_i,v_{i+1})$-path.  Because each $P_i$ has length at most $k$ and $C$ is an induced cycle in $D_{\leq k}$, it follows that $d(v_i,v_{i-2}) > k$.  This means by Lemma \ref{Lemma-IntersectingVertex} that for each $i \in [1,\ell-2]$, $u_{i-1,i}$ is the unique vertex intersecting $P_{i-1}$ and $P_i$ such that $N(u_{i-1,i}) \cap (P \triangle Q) \neq \emptyset$.   Observe that $u_{i-1,i}, u_{i,i+1} \in P_{i}$ when $i \in [1,\ell-3]$, and also $u_{0,1} \in P_0$ and $u_{\ell-3,\ell-2} \in P_{\ell-2}$.  So, $P_0$ intersects $P_1$, $P_{\ell-2}$ intersects $P_{\ell-3}$, and for each $i \in [1,\ell-3]$, $P_i$ intersects both $P_{i-1}$ and $P_{i+1}$.%, we show that no other path intersections occur.  Since $T$ is a tree, $P_0$ is the $(v_0,v_1)$-path, and since $P_1$ is the unique $(v_1,v_2)$-path, it clearly intersects $P_0$
		
		%It is sufficient to show that $P_0$ intersects with only $P_1$, and similarly $P_{\ell-2}$ only with $P_{\ell-3}$, and also for each $i \in [1,\ell-3]$, $P_i$ intersects with exactly $P_{i-1}$ and $P_{i+1}$.  Suppose for some $j \in [2,\ell-2]$ $P_0$ intersects with $P_j$.  Then 
		
		It is sufficient to show that there are no other path intersections.  %Notice that each $P_i$ can intersect either $P_{i-1}$ or $P_{i+1}$ and no other path.  
		Suppose otherwise that $P_i$ intersects $P_j$ for some $j \in [0,\ell-2] \setminus \{i-1,i+1\}$.  Then if $j > i+1$, for each $r \in [i+1,j-1]$, $P_r$ intersects $P_{r+1}$, so $T$ contains two paths between $v_{i+1}$ and $v_j$, namely $$(P_{i+1}, P_{i+2}, \ldots, P_{j-1}), \text{ and } (v_j, \ldots, u_{j-1,j}, \ldots, u_{j,i}, \ldots, v_{i+1}).$$
		This means that $T$ has a cycle, a contradiction.  The argument is analogous for $j < i-1$. \qedhere
		
		%Let $u_{i,i-1}$ and $u_{i,i+1}$ be the intersecting vertices for $P_{i-1}$ and $P_{i+1}$, respectively.  Let $a^- = d(v_{i-1},u_{i,i-1})$ and $a^+ = d(v_{i+1},u_{i,i+1})$.  Then since $d(v_i,v_{i-1}) = d(v_i,v_{i+1}) = k$ and $d(v_i,v_j) > k$, it holds that $d(v_i,u_{i,i-1}) = k-a^-$ and $d(v_i,u_{i,i+1}) = k- a^+$.
	\end{proof}
	
	\begin{prop}
		The graph $D_{\leq k}$ is chordal.
	\end{prop}
	\begin{proof}
		Let $C = (v_0,v_1, \ldots, v_{\ell-1},v_0)$ be an induced cycle of $D_{\leq k}$ satisfying $\ell \geq 4$.  
		
		[***We know what to do when $\ell = 4$***]
		
		Suppose $\ell \geq 5$.  By Lemma \ref{Lemma-WheelCase}, the subgraph of $T$ induced by the consecutive $(v_i,v_{i+1})$-paths $P_0,P_1, \ldots, P_{\ell-2}$ satisfies the property that only consecutive paths intersect. Denote the intersecting vertex for $P_{i-1}$ and $P_{i}$ with the properties in Lemma \ref{Lemma-IntersectingVertex} by $u_{i-1,i}$.  For $i \in [1,\ell-1]$, let $a_i := d(v_{i}, u_{i-1,i})$, and for $i \in [2,\ell-2]$, $s_i := d(u_{i-2,i-1}, u_{i-1,i})$.  For $i = 1$, define $s_1 := d(v_0,u_{0,1})$ and similarly for $i = \ell-1$, define $s_{\ell-1} := d(u_{\ell-3,\ell-2}, v_{\ell-1})$.
		
		Note that $d(v_i,v_{i+1}) = a_{i-1} + s_{i} + a_{i}$ when $i \in [2,\ell-2]$, and $d(v_0,v_1) = s_1 + a_1$ and $d(v_{\ell-2},v_{\ell-1}) = a_{\ell-2}+s_{\ell-1}$.  Observe that the unique $(v_0,v_{\ell-1})$-path of length at most $k$ is the path along the entire spine of the graph induced by $\{P_i: i \in [0,\ell-2]\}$, which contains the subsequence of vertices $(v_0, \ldots, u_{0,1}, \ldots, u_{1,2}, \ldots, u_{\ell-3,\ell-2}, \ldots, v_{\ell-1})$ and has length $\sum_{i=1}^{\ell-1} s_i$.  Thus we have
			\begin{align*}
				d(v_0,v_{\ell-1}) =  \sum_{i=1}^{\ell-1} s_i \leq k	
				%d(v_0,v_{\ell-1}) &\leq  (k-a_1) + (k- a_{\ell-2}) + \sum_{i=1}^{\ell-2} k - a_i - a_{i-1}	\\
				%&\leq \ell k - \sum_{i = 1}^{\ell-2} 2a_i,
			\end{align*}
		But note that for all $j \in [2, \ell-2]$, $k < d(v_0,v_j) = a_j + \sum_{i = 1}^js_j$; and similarly, for $j \in [1, \ell-3]$, $k < d(v_j, v_{\ell-1}) = a_j + \sum_{i = j+1}^{\ell-1}s_j$.  So, in particular, choose any $j' \in [2,\ell-3]$, which is nonempty because $\ell \geq 5$.  Then
		\begin{align*}
			2k &< d(v_0,v_{j'}) + d(v_{j'},v_{\ell-1}) 	\\
			&= 2a_{j'} + \sum_{i = 1}^{\ell-1}s_i 	\\
			&= 2a_{j'} + d(v_0,v_{\ell-1}) \leq 2a_{j'} + k.
		\end{align*}
		Thus we have $2a_{j'} > k$.  However, since 
		\begin{align*}
			k &< d(v_{j'-1},v_{j'+1}) 	\\
			&\leq a_{j'-1} + (k-a_{j'-1}-a_{j'}) + (k - a_{j'}-a_{j'+1}) + a_{j'+1} 	\\
			&= 2k - 2a_{j'},
		\end{align*}
		it follows that $2a_{j'} < k$, a contradiction.  We have shown that $D_{\leq k}$ has no induced cycle of length at least $4$, which means that it is chordal. \qedhere
		%which is equivalent to $(\ell-1)k \leq \sum_{i = 1}^{\ell-2} 2a_i$.  But since $d(v_{i-1},v_{i+1}) > k$ for each $i \in [1,\ell-2]$, and $d(v_{i-1},v_{i+1}) = 2k - 2a_i$, it holds that 
		%$$2k-2a_i > k \Leftrightarrow k > 2a_i.$$
		%Thus $(\ell - 1)k = \sum_{i = 1}^{\ell-2} 2a_i <(\ell-2)k$, a contradiction.
	\end{proof}
\end{document}